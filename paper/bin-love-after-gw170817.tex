\documentclass[prd,twocolumn,nofootinbib,superscriptaddress,amsmath,amssymb]{revtex4-1}
\usepackage{mathtools}
\usepackage{graphics}
\usepackage{graphicx}
\graphicspath{{./images/}}
\usepackage{dcolumn}
\usepackage{bm}
\usepackage{dsfont} 
\usepackage{amsmath,amssymb}
\usepackage{hyperref}
\usepackage{tabularx}
\usepackage{epstopdf}
\usepackage[normalem]{ulem}
\usepackage[usenames]{color}
\usepackage{multirow}
\usepackage{makecell}
\usepackage{diagbox}
\epstopdfsetup{outdir=./images/}
\allowdisplaybreaks
\hypersetup{
    colorlinks=true,
    linkcolor=blue,
    filecolor=magenta,      
    urlcolor=blue,
    citecolor=blue
}
\urlstyle{same}

\newcommand{\red}[1]{\protect\color{red} #1 \protect\color{black}}
\newcommand{\green}[1]{\protect\color{green} #1 \protect\color{black}}
\newcommand{\blue}[1]{\protect\color{blue} #1 \protect\color{black}}
\newcommand{\black}[1]{\protect\color{black} #1 \protect\color{black}}
\newcommand{\yellow}[1]{\protect\color{yellow} #1 \protect\color{black}}
\newcommand{\bra}[1]{\protect\langle #1 |}
\newcommand{\ket}[1]{| #1 \protect\rangle}
\newcommand{\braket}[2]{\protect\langle #1 | #2 \protect\rangle}
\newcommand{\expected}[1]{\protect\langle #1 \protect\rangle}
\newcommand{\R}{{\mbox{\tiny R}}}
\newcommand{\ky}[1]{\textcolor{blue}{\it{\textbf{ky: #1}}} }
\newcommand{\zc}[1]{\textcolor{red}{\it{\textbf{zc: #1}}} }
\definecolor{red(ncs)}{rgb}{0.77, 0.01, 0.2}
\newcommand{\ny}[1]{\textcolor{blue}{NY: #1} }
\newcommand{\kc}[1]{\textcolor{green}{KC: #1} }
\newcommand{\NS}{{\mbox{\tiny NS}}}

\usepackage{array}
\newcolumntype{C}[1]{>{\centering\let\newline\\\arraybackslash\hspace{0pt}}m{#1}}


\newcolumntype{C}[1]{>{\centering\arraybackslash}m{#1}}

\def\eq#1{Eq.~(\ref{eq:#1})}
\def\Eq#1{Equation~(\ref{eq:#1})}
\def\eqs#1#2{Eqs.~(\ref{eq:#1}) \& (\ref{eq:#2})}
\def\eqlist#1#2{Eqs.~(\ref{eq:#1}-\ref{eq:#2})}
\def\Eqs#1#2{Equations~(\ref{eq:#1}) \& (\ref{eq:#2})}
\def\Eqlist#1#2{Equations~(\ref{eq:#1}-\ref{eq:#2})}
\def\fig#1{Fig.\ref{fig:#1}}
\def\figs#1#2{Figs.\ref{fig:#1} \& \ref{fig:#2}}
\def\Fig#1{Figure~\ref{fig:#1}}
\def\Figs#1#2{Figures~\ref{fig:#1} \& \ref{fig:#2}}
\def\tab#1{Table~\ref{tab:#1}}
\def\sec#1{Section~\ref{sec:#1}}




\begin{document}

\title{EoS-insensitive relations after GW170817}

\author{Zachary Carson}
\affiliation{%
 Department of Physics, University of Virginia, Charlottesville, Virginia 22904, USA
}%

\author{Katerina Chatziioannou}
%\affiliation{%
% Canadian Institute for Theoretical Astrophysics, 60 St. George Street, Toronto, Ontario, M5S 3H8, Canada
%}%
\affiliation{%
 Center for Computational Astrophysics, Flatiron Institute, 162 5th Ave, New York, NY 10010
}%

\author{Carl-Johan Haster}
\affiliation{%
 Canadian Institute for Theoretical Astrophysics, 60 St. George Street, Toronto, Ontario, M5S 3H8, Canada
}%
\affiliation{%
 Center for Computational Astrophysics, Flatiron Institute, 162 5th Ave, New York, NY 10010
}%

\author{Nicol\`as Yunes}
\affiliation{%
 Department of Physics, Montana State University, Bozeman, MT 59717, USA.
}%
\affiliation{%
 Department of Physics and MIT Kavli Institute,
}%

\author{Kent Yagi}
\affiliation{%
 Department of Physics, University of Virginia, Charlottesville, Virginia 22904, USA
}%


\date{\today}

%%%%%%%%%%%%%%%%%%%%%%%%%%%%%%%% Begin Abstract %%%%%%%%%%%%%%%%%%%%%%%%%%%%%%%%%%%%%%%%%%%%%%%%%%%%%%%%%%%%%%%%%%%%%%%%%%%%%%%%%%%%%%%%%%%%%%%%%%%%%%%%%%%%%%%%%%%%%%%%%%%%%%%%%%%%%%%%%%%%%%%%%%%%%%%%%%%%%
\begin{abstract}
% Intro
The thermodynamic relationship between pressure and density (the equation of state, or EoS) of cold supranuclear matter, accessible in neutron stars, is critical to the study of such stars, and is one of the largest uncertainties in nuclear physics to date. 
% How to probe
The extraction of tidal deformabilities from the gravitational waveforms of binary neutron star merger events, such as GW170817, is a promising method of probing the nuclear structure.
% What was done before - Binary Love
Previous studies have shown that approximately EoS-insensitive ``binary Love relations" exist between symmetric and antisymmetric combinations of individual tidal deformabilities, which are difficult to be independently measured with second-generation gravitational wave interferometers.
% What was done before - I-Love-Q
Similarly, another set of EoS-insensitive relations exist between individual neutron star parameters: moment of inertia (I), tidal deformability (Love number), quadrupole moment (Q), and compactness (C) - known as the ``I-Love-Q" and ``C-Love" relations.
% What universal relations do
Such EoS-insensitive relations allow the elimination of some tidal parameters from the list of model parameters, thus breaking degeneracies and ultimately reducing parameter extraction uncertainties.
% What we will do
However, even the most precise of EoS insensitive relations carry EoS variation which inject systematic errors into the parameter extraction process.
In this document, we explain how one can reduce equation of state variation for I-Love-Q, C-Love, and binary Love relations, which helps to reduce systematic errors on the tidal measurement of binary neutron star mergers.
However, we show that the systematic effects of these relations is very small until next-generation detectors. 
We achieve this  by restricting to only equations of state drawn from the 90\% posterior constraint on pressure as a function of density, as derived by the LIGO Collaboration.
% What we found
We find an improvement in binary Love universality by a factor of $\sim 59$\% for stars with a mass ratio of 0.75, in C-Love universality by a factor of $\sim 67\%$, and in I-Love-Q universality by factors of $\sim 50$\%.
% Conclude
We end by comparing systematic errors on tidal measurement due to the equation of state variation with statistical errors and comment on whether one can safely use these EoS-insensitive relations with future gravitational wave observations.
We conclude that the systematic errors injected from the use of improved EoS-insensitive relations become comparable to the statistical errors for future design upgrade Voyager.
\end{abstract}

\maketitle

%%%%%%%%%%%%%%%%%%%%%%%%%%%%%%% Begin Introduction %%%%%%%%%%%%%%%%%%%%%%%%%%%%%%%%%%%%%%%%%%%%%%%%%%%%%%%%%%%%%%%%%%%%%%%%%%%%%%%%%%%%%%%%%%%%%%%%%%%%%%%%%%%%%%%%%%%%%%%%%%%%%%%%%%%%%%%%%%%%%%%%%%%%%%%%%%%%%%

\section{Introduction}\label{sec:intro}
%Intro

{\ny{Just some general comments so that we don't forget:
\begin{itemize}
\item The EoS-insensitive relations are called ``binary Love'', ``I-Love-Q'', ``C-Love'' and ``R-Love.'' There is no need to call them ``binary Love EoS-insensitive relations.'' That's too verbose and not what they are called in the literature. I am ok with describing these relations as EoS-insensitive instead of approximately universal, but let's not change the names. 
\item There is some serious tidying up that needs to be done to all of the text. Specifically, there are missing prepositions, like ``the'', in lots of places that makes the reading a bit choppy and we can improve. Zack, when you have a chance, please do a detail read through and fix the flow as much as you can. 
\item Eventually, we need to discuss the abstract and the intro in more detail. 
\item Let's not forget to check the acknowledgments for funding sources. 
\item Let's do a spell check. 
\item Let's remove all commented out text.  
\end{itemize}
}}
The dependence of pressure on density for cold supranuclear matter found inside neutron stars (NSs) remains to be one of the largest mysteries in both nuclear physics and astrophysics.
This internal structure of such stars, known as the equation of state (EoS), is exceedingly important, as it is the determining factor for many NS observables - such as the mass, the radius, and many more.
Unfortunately, terrestrial experiments can only study the EoS to around the nuclear saturation density ($\rho_{\text{sat}} \approx 2.5 \times 10^{14} \text{ g/cm}^3$)~\cite{Li:HeavyIon,Tsang:SymmetryEnergy,Centelles:NeutronSkin,Li:CrossSections,Chen:SymEnergy} (although some temperature-dependent heavy-ion collision experiments may probe higher densities~\cite{Danielewicz:2002pu}), making NSs ideal laboratories for constraining ultra dense nuclear matter.

%Constraining the EoS
Independent measurements of NS observables can be used to constrain the nuclear EoS.
For example, electromagnetic observations of the mass-radius relationship of NSs have been used to place limits on the NS EoS~\cite{guver,ozel-baym-guver,steiner-lattimer-brown,Lattimer2014,Ozel:2016oaf}.
However, these implications potentially suffer from large systematic errors due to various x-ray burst astrophysical mismodeling uncertainties.
Alternatively, the emission of gravitational waves (GWs) from binary NS merger events have proven to be a unique method of probing nuclear physics.
During the early inspiral portion of the merger, the orbital separation is large enough that the effect of companion tidal fields is negligible. 
As the separation decreases through GW emission, tidal forces are magnified and the NSs deform from sphericity, altering the orbital trajectories; a relic which is directly imprinted on the GW signal.
This deformation is characterized by the \textit{tidal deformability} $\Lambda$ as the linear response of the NSs quadrupole moment $Q_{ij}$ to the neighboring tidal field $\varepsilon_{ij}$~\cite{hinderer-love,Flanagan2008}.

%reparameterization of the template
In the case of binary NSs, each star becomes deformed via the process highlighted above, resulting in two highly correlated tidal parameters $\Lambda_1$ and $\Lambda_2$ entering in the gravitational waveform\cite{Flanagan2008,Vines:2011ud}.
Due to high correlations between between the two parameters, independent extraction becomes very difficult at current interferometer sensitivities~\cite{Wade:tidalCorrections}.
Typically, these high correlations can be reduced by strategically reparameterizing the waveform template with new tidal parameters constructed from linear combinations of $\Lambda_1$ and $\Lambda_2$.
For example, utilizing the parameters $\tilde{\Lambda}$ and $\delta \tilde{\Lambda}$~\cite{Favata:2013rwa,Wade:tidalCorrections} which enter the GW waveform at different post-Newtonian (PN) orders (powers of the ratio between the orbital speed and the speed of light $v/c^{2n}$) partially mitigates these correlations.
Similarly, the waveform may be parameterized by mass-independent tidal parameters $\lambda_0$ and $\lambda_1$, derived through a Taylor expansion about $m_0=1.4 \text{ M}_{\odot}$ of tidal deformability $\Lambda \approx \lambda_0+\lambda+1(1-\frac{m}{m_0})$~\cite{delPozzo:TaylorTidal,Yagi:binLove}. 
Use of this set of tidal parameters allows one to effectively combine multiple events with varying mass and $\tilde\Lambda$ components, not possible in the former parameterization of $\tilde{\Lambda}$ and $\delta \tilde{\Lambda}$

%Universal relations
Unfortunately, current detectors are only sensitive enough to accurately measure the dominant tidal parameter, $\lambda_0$ (also known as $\Lambda_{1.4}$, the tidal deformability at $1.4 \text{ M}_{\odot}$).
Previous work by Yagi and Yunes~\cite{Yagi:binLove} resolved this issue by finding approximately EoS-insensitive ``binary Love relations" between symmetric and anti-symmetric combinations of tidal deformabilities $\Lambda_{s,a}=\frac{1}{2}(\Lambda_1 \pm \Lambda_2)$.
This allows one to further break degeneracies between tidal parameters, resulting in both (i) a decrease in uncertainty upon parameter extraction of $\Lambda_s$ on the removal of parameter $\Lambda_a$ from the model, and (ii) the ability to measure the secondary tidal parameter upon measurement of the first~\cite{Katerina:residuals}.  
Use of these relations leads to up to an order of magnitude improvement in parameter estimation of $\Lambda_s$ through a simple Fisher analysis~\cite{Yagi:binLove}, also confirmed with the detailed analysis of~\cite{Katerina:residuals}.
Similar ``I-Love-Q" and ``C-Love" relations~\cite{Yagi:ILQ} have been found between individual NS observables: quadrupole moment , moment of inertia, tidal deformability, and compactness; similarly allowing the removal of tidal parameters from the model list, and allowing an automatic determination of one observable through the measurement of another.
However, these relations have been tuned to a collection of EoS models, and each have intrinsic uncertainties associated with their use.

%How we improved this
In this analysis, we aim to improve this important work by imposing the restrictions found on the EoS~\cite{LIGO:posterior} derived from the recent binary NS merger observation, GW170817~\cite{TheLIGOScientific:2017qsa}.
By restricting to only EoSs which lie within the the 90\% credible region on pressure as a function of density found in~\cite{LIGO:posterior}, we repeat the analyses done in~\cite{Yagi:binLove,Yagi:ILQ} and show an increase in universality for I-Love-Q, C-Love, and binary Love relations.
We also investigate the relationship between the NS radius and its tidal deformability, or the R-Love relations, determining if the uncertainties can be reduced.
In addition, we consider hybrid star EoSs seen in~\cite{Paschalidis2018}, which experience strong first-order transitions from hadronic matter to quark matter.
These provide a departure from the nuclear matter relations considered here - providing valuable insight into them.
Further, we estimate the viability of using improved EoS-insensitive relations on future GW detections by performing simple Fisher analyses to approximate when the statistical errors from parameter extraction become comparable to the systematic errors due to EoS variation in the relations.
To estimate the effect of waveform mismodeling, we additionally consider two different tidal corrections to the injected PhenomD~\cite{PhenomDI,PhenomDII} waveform: 6PN~\cite{Wade:tidalCorrections} and NRTidal~\cite{Samajdar:NRTidal}

\subsection{Executive Summary}
%What we did
In this paper we attempt to improve I-Love-Q, C-Love, and binary Love relations by imposing constraints on the EoS found by GW170817~\cite{LIGO:posterior,TheLIGOScientific:2017qsa}.
First, we generate two large samples of spectral EoSs~\cite{Lindblom:2018rfr}, where we (i) impose the restriction that they must be drawn from GW170817s 90\% credible posterior in pressure as a function of density, and (ii) don't impose any restrictions for comparison purposes.
Next, we follow the important works of~\cite{Yagi:binLove} and~\cite{Yagi:ILQ} to show how the ``constrained" set of EoSs show improved universality (binary Love, I-Love-Q, and C-Love) from both the previous works, and the ``unconstrained" sample, shown by Fig.~\ref{fig:ILQ}.
In addition, we determine the degree to which hybrid star EoSs obey binary Love relations.
Finally, through a simple Fisher analysis, we consider the value in improving EoS-insensitive relations by approximating when the statistical errors from parameter extraction become comparable to the systematic errors due to EoS variation in the relations.

%Final Results
Upon use of the constrained set of EoSs, we find an improvement in binary Love universality by a factor of $\sim 59$\% for stars with mass ratios of $0.75$, in C-Love universality by a factor of $\sim 67\%$, and in I-Love-Q universality by factors of $\sim50$\%, tabulated in Tab.~\ref{tab:maxVar}.
We also present the relationship between the NS radius and tidal deformability, computed directly from the C-Love relations, showing maximum EoS variability of $358.9 \text{ km}$ from the fit discussed in Sec.~\ref{sec:clove}.
Additionally, we find that hybrid quark-hadron stars do obey the I-Love-Q and C-Love relations, with sightly increased EoS variability.
We also find that binaries consisting of one massive hybrid star, and one small-mass hadronic star do not agree with derived binary Love relations, due to the large separation in $\tilde{\Lambda}$ between the constituent stars.
Due to the current limitations in detector sensitivity, the systematic errors arising from EoS variation in the EoS-insensitive relations are far outweighed by the statistical errors accrued by extraction of high-order PN tidal parameters from GW waveforms.
Fig.~\ref{fig:stackedFisher} compiles the results from Fisher analyses approximating the uncertainties accrued from the parameter extraction of $\lambda_0$ (the tidal deformability at $1.4 \text{ M}_{\odot}$) for GW170817 as if it had been observed by future detectors $A \equiv ($ O2~\cite{aLIGO}, aLIGO~\cite{aLIGO}, A\texttt{+}~\cite{Ap_Voyager_CE}, Voyager~\cite{Ap_Voyager_CE}, ET~\cite{ET}, CE~\cite{Ap_Voyager_CE}$)$.
Here, $\sigma^A_{\text{GW170817}}$ and $\rho^A_{\text{GW170817}}$ correspond to the statistical error and signal-to-noise-ratio found from observing GW170817 on detector $A$, and $\sigma^A_N$ represents the combined uncertainty after the detection of $N_A$ events (corresponding to the binary NS merger rate associated interferometer $A$).
These values are further tabulated in Table~\ref{tab:variances}.
It can be seen from this figure that the statistical errors from parameter extraction become comparable the EoS-insensitive relation systematics (indicated by the magenta horizontal line) for the Voyager telescope - indicating when improved binary Love relations will become necessary.
Also shown in the figure is the same measurement uncertainties upon the injection of the PhenomD + NRTidal waveform, rather than the PhenomD + 6PN tidal correction to demonstrate the effect of waveform mismodeling.
\begin{figure}
\begin{center} 
\includegraphics[width=\columnwidth]{stackedFisher.eps}
\end{center}
\caption{
Estimated statistical uncertainties on the extraction of the tidal deformability at $1.4 \text{ M}_{\odot}$, $\lambda_0$, observed from interferometers $A=$(O2, aLIGO, A\texttt{+}, Voyager, CE, ET-D$)$ as a function of the signal-to-noise-ratio.
In this figure, the filled blue circles denote the single-event statistical uncertainties $\sigma^A_{\text{GW170817}}$ of GW170817 as if it were observed on detector $A$.
Also shown by the turquoise shaded region is combined uncertainty $\sigma^A_N$  from the observation of multiple events from a simulated population of size $N_A$ corresponding to the upper and lower limits of the expected binary NS merger detection rates for O2, aLIGO, A\texttt{+}, Voyager, CE, and ET-D over 1 observation year.
It can be seen that the statistical errors in the parameter extraction become comparable to the systematics from EoS variation in the improved binary Love relations (indicated by the horizontal indigo dashed line) for Voyager, CE, and ET.
These results are further tabulated in Table~\ref{tab:variances}.
Additionally shown in the figure are the estimated statistical uncertainties on the extraction of $\lambda_0$ upon the injection of the PhenomD + NRTidal correction waveform (IMRD + NRTidal), rather than the PhenomD + 6PN tidal correction (IMRD + 6PN), in order to demonstrate the effect of waveform mismodeling uncertainties.
The single-detection uncertainties are shown by unfilled maroon circles, while the combined detection uncertainties are outlined by dashed maroon.
}
\label{fig:stackedFisher}
\end{figure} 

%outline
The organization of this paper is outlined below.
We begin with a complementary background and theory material in Sec.~\ref{sec:theory}.
We continue in Sec.~\ref{sec:universal} by finding new and improved binary Love, I-Love-Q, and C-Love relations, and considering how well hybrid star EoSs agree with the EoS-insensitive relations.
We next examine these improved relations and question whether or not they are useful for future interferometers in Sec.~\ref{sec:observations}.
We conclude in Sec.~\ref{sec:conclusion} by discussing our results and mentioning avenues of future work.
Throughout this paper, we have adopted geometric units of $G=c=1$, unless otherwise stated.

%%%%%%%%%%%%%%%%%%%%%%%%%%%%%%%%% Begin theory %%%%%%%%%%%%%%%%%%%%%%%%%%%%%%%%%%%%%%%%%%%%%%%%%%%%%%%%%%%%%%%%%%%%%%%%%%%%%%%%%%%%%%%%%%%%%%%%%%%%%%%%%%%%%%%%%%%%%%%%%%%%%%%%%%%%%%%%%%%%%%%%%%%%%%%%%%%%
\section{Background and theory}\label{sec:theory}

\subsection{Spectral representations of neutron star equations of state}\label{sec:eos}
The structure of a NS and its tidal interactions in a binary system rely heavily on the underlying state function (or equation of state) describing the relationship between pressure ($p$) and energy density ($\epsilon$) of nuclear matter.
Given that all currently proposed EoSs utilize certain approximations~\cite{Oertel:Review,Baym:Review}, one method to study a wide range of physically realizable EoSs is to parameterize them such that any realistic EoS can be represented with a small number of parameters.
Spectral representations~\cite{Lindblom:2010bb,Lindblom:2012zi,Lindblom:2013kra,Lindblom:2018rfr,Abbott:2018exr} parameterize EoSs by performing spectral expansions on the adiabatic index $\Gamma(p)$\footnote{Another way of parameterizing EoSs is the piecewise polytropic formulation~\cite{Read2009,Lackey:2014fwa,Carney:2018sdv}.}:
\begin{equation}
\Gamma(x) = \exp{\sum_k\gamma_k x^k},
\end{equation}
where $x \equiv \log{p/p_0}$ for a minimum pressure $p_0$.
The equation of state is then determined by an integration of the differential equation:
\begin{equation}
\frac{d \epsilon(p)}{dp}=\frac{\epsilon(p)+p}{p \Gamma(p)}.
\end{equation}
Using this formalism, any valid EoS can be approximated through the choice of 4 or more spectral coefficients $\gamma_k$, tabulated for several common EoSs in Table 1 of~\cite{Lindblom:2018rfr}.

In the current analysis, we consider EoSs which have been constrained by the recent binary neutron star merger event, GW170817.
Important recent work~\cite{LIGO:posterior,Carney:2018sdv} sampled EoS parameters in order to derive a marginalized posterior on the pressure as a function of mass density, as seen in Fig. 2 of~\cite{LIGO:posterior}.
The spectral coefficients were sampled within the ranges: $\gamma_0 \in \lbrack 0.2,2 \rbrack$, $\gamma_1 \in \lbrack -1.6,1.7 \rbrack$, $\gamma_2 \in \lbrack -0.6,0.6 \rbrack$, $\gamma_0 \in \lbrack -0.02,0.02 \rbrack$, and further restricted the adiabatic index to be $\Gamma \in \lbrack 0.6,4.5 \rbrack$, ensuring the parameterization exposed a wide range of viable EoSs~\cite{Lindblom:parameters}.
Additional constraints imposed upon the generated EoSs were as follows~\cite{LIGO:posterior}: (i) causality within 10\%, and (ii) EoS priors must support NS masses up to $1.97 \text{ M}_{\odot}$, consistent with astrophysical observations~\cite{Zhao:massiveNS}.
We utilize a random set of 100 of the above posterior samples defined by GW170817, furthermore referenced as the ``constrained EoSs", as shown in Fig.~\ref{fig:eos}.
Following~\cite{Read2009}, the parameterized high-density core EoSs generated above are matched to the low-density crust EoS Sly~\cite{Douchin:2001sv} at about half of the nuclear saturation density, $\rho_{\text{stitch}}=1.3 \times 10^{14} \text{ g/cm}^3$.
For comparison, we also include a second sample of 100 ``unconstrained" EoSs, randomly sampled in the pressure-density plane to be unrestricted by GW170817s posterior.

In addition, we investigate 10 transitional quark-hadron matter stars, which undergo first-order phase transitions at pressure $P_{\text{tr}}$, where the hadronic branch departs into a quark-matter branch at a given mass transition~\cite{Paschalidis2018,Alford:2017qgh,1971SvA....15..347S,Zdunik:2012dj,Alford:2013aca}.
In particular, we focus on the ACS and ACB models described in Ref.~\cite{Paschalidis2018}, shown in Fig.~\ref{fig:hybridML}.
These result in two distinct types of NSs, based on their observed mass: (i) massive ($M \geq \text{ M}_{\text{tr}}$) hybrid stars which have quark-matter cores and nuclear matter crusts (we denote this as ``HS"), and (ii) low-mass ($M \leq \text{ M}_{\text{tr}}$) hadronic stars with no internal transition to quark matter (we denote this as ``NS").

\begin{figure}
\begin{center} 
\includegraphics[width=\columnwidth]{EoSs.eps}
\end{center}
\caption{
Small representative sets of the unconstrained (dotted) and constrained (solid) EoSs used in this analysis. 
The constrained EoSs were populated by randomly selecting 100 posterior samples obtained from the GW170817 90\% credible level shown in cyan~\cite{LIGO:posterior}.
Additionally, 100 unconstrained EoSs were generated by randomly sampling spectral coefficients of physically valid EoSs, with no further constraints applied.
The following restrictions were also applied to both sets of EoSs: (i) the EoS must have a causal structure within 10\%, and (ii) the EoS must support maximum NS mass of at least $1.97 \text{ M}_{\odot}$, consistent with astrophysical observations.
Finally, the high density core EoSs generated above were stitched to the low-density crust EoS of SLy~\cite{Douchin:2001sv} at $\rho_{\text{stitch}}=1.3 \times 10^{14} \text{ g/cm}^3$.}
\label{fig:eos}
\end{figure} 

\begin{figure}
\begin{center} 
\includegraphics[width=\linewidth]{hybridML.eps}
\end{center}
\caption{
$M-\Lambda$ relations for the ACS and ACB class model hybrid stars. 
Observe the transitional points where the NS transitions from the hadronic branch into a quark-matter branch.
As described in the text, HS/HS and NS/NS binaries have individual masses chosen such that both stars are contained within one branch -- resulting in high $q$ values.
Constrastingly, HS/NS binaries have individual masses chosen from each branch resulting in small $q$-values.
}
\label{fig:hybridML}
\end{figure} 

\subsection{Neutron Star Tidal deformability}\label{tidal}
Binary neutron star mergers, such as GW170817, provide valuable insight into the internal structure of such stars. 
Namely, tidal effects which enter the gravitational waveform depend strongly on the EoS defining the NSs structure. 
The ($l=2$ electric type) \emph{tidal deformability}, denoted as $\lambda$, has the largest impact on the waveform phase. 
Consider a NS of mass $M$ in the presence of the tidal field $\varepsilon_{ij}$ of a companion star.
In response, the NS will deform away from sphericity under the acquisition of quadrupole moment $Q_{ij}$, characterized by $\lambda$ as the linear response to $\varepsilon_{ij}$~\cite{Flanagan2008,hinderer-love,Yagi2013}:
\begin{equation}
Q_{ij}=-\lambda \varepsilon_{ij}.
\end{equation}
One can extract $Q_{ij}$ and $\varepsilon_{ij}$ -- and thus $\lambda$, or its dimensionless form $\Lambda \equiv \lambda/M^5$ -- via different asymptotic limits of the gravitational potential:
\begin{align}
\begin{split}
\Phi=\frac{1+g_{tt}}{2}&=-\frac{M}{r} - \frac{3}{2}\frac{Q_{ij}}{r^3} \Bigg(\frac{x^i}{r} \frac{x^j}{r}-\frac{1}{3}\delta_{ij} \Bigg) + \mathcal{O} \Bigg( \frac{M^4}{r^4} \Bigg)\\
&+ \frac{1}{2} \varepsilon_{ij} x^i x^j + \mathcal{O} \Bigg( \frac{r^3}{M^3} \Bigg).
\end{split}
\end{align}
The metric component $g_{tt}$ is determined by first constructing a spherically symmetric, non-spinning background solution, followed by the introduction of a perturbative tidal deformation.
Next, the perturbed Einstein equations must be solved in the NSs interior given an EoS, then matched to the exterior Schwarzschild solution at the surface, modulo a constant.
This process is described in more detail in Ref.~\cite{hinderer-love}.
Further, the stellar mass and radius are determined from $p(R)=0$, and the above mentioned constant.
Similarly, the moment of inertia is obtained from the asymptotic behavior of the $g_{t \phi}$ metric component.

Next we consider binary NS systems, as was found in GW170817, where each star individually experiences a neighboring tidal field.
Thus each star possesses tidal deformabilities $\Lambda_1$ and $\Lambda_2$ entering the gravitational waveform.
Due to strong correlations between these parameters, individual extraction is very difficult with current interferometer sensitivity limitations.
However, the waveform templates can be strategically reparameterized to instead include independent linear combinations of $\Lambda_1$ and $\Lambda_2$ to mitigate correlations between them. 
Any set of two independent functions may be used, for example: tidal parameters $\tilde{\Lambda}=\tilde{\Lambda}(\Lambda_1,\Lambda_2)$ and $\delta \tilde{\Lambda}=\delta \tilde{\Lambda}(\Lambda_1,\Lambda_2)$~\cite{Wade:tidalCorrections} first enter the gravitational waveform at 5PN and 6PN orders respectively, thus partially breaking degeneracies.
Here, $\tilde{\Lambda}$ is known as the mass weighted tidal deformability and is the dominant tidal parameter in the waveform. 

\subsection{EoS insensitive relations}\label{sec:eosInsensitive}
Current gravitational wave interferometry is not yet sensitive enough to accurately extract both tidal parameters $\tilde{\Lambda}$ and $\delta\tilde{\Lambda}$.
In a search to remedy this, Yagi and Yunes~\cite{Yagi:binLove} found that symmetric and antisymmetric combinations of tidal deformabilities:
\begin{equation}
\Lambda_s \equiv \frac{\Lambda_1 + \Lambda_2}{2}, \hspace{6mm} \Lambda_a \equiv \frac{\Lambda_1 - \Lambda_2}{2},
\end{equation}
display EoS-insensitive properties to a high degree, showing EoS variations up to 20\% for binaries with masses less than $1.7 \text{ M}_{\odot}$, for a representative set of 11 EoSs. 
These ``binary Love relations" allow one to break degeneracies between coupled tidal parameters.
This is important for two reasons: (i) they allow one to algebraically eliminate tidal parameters from the template parameter list, improving overall parameter estimation, and (ii) they allow the automatic measurement of a second tidal parameter given observation of the first.
It was shown through a simple Fisher analysis that the use of binary Love relations improved parameter extraction on $\tilde{\Lambda}$ by up to an order of magnitude.

Similar EoS-insensitive relations have been found to exist between individual NS observables: moment of inertia (I), tidal deformability (Love), quadrupole moment (Q), and compactness (C), known as the ``I-Love-Q" and ``C-Love" relations~\cite{Yagi:ILQ, Yagi:binLove}.
These have been found to be EoS insensitive by up to 1\% and 6.5\% respectively.
In this paper, we show improvement in all of these relations by restricting to EoSs sampled from GW170817s posterior on pressure as a function of density.
%%%%%%%%%%%%%%%%%%%%%%%%% Begin universal relations %%%%%%%%%%%%%%%%%%%%%%%%%%%%%%%%%%%%%%%%%%%%%%%%%%%%%%%%%%%%%%%%%%%%%%%%%%%%%%%%%%%%%%%%%%%%%%%%%%%%%%%%%%%%%%%%%%%%%%%%%%%%%%%%%%%%%%%%%%%%%%%%%%%%%
\section{EoS-insensitive relations}\label{sec:universal}
In this section, we follow the analyses performed in~\cite{Yagi:binLove,Yagi:ILQ} with two new sets of 100 spectrally generated EoS: (i) posterior samples randomly selected from GW170817s 90\% posterior on on pressure as a function of density as shown by the solid green curves in Fig.~\ref{fig:eos}, and (ii) those unconstrained by any prior EoS information, as shown by the dotted maroon curves in Fig.~\ref{fig:eos}.

\subsection{I-Love-Q relations}\label{sec:ilq}
Here we present our results on I-Love-Q universality as compared to Fig. 1 of Ref.~\cite{Yagi:ILQ}.
In particular, we consider two distinct classes of NSs: nuclear matter EoSs and hybrid quark-hadron star EoSs as described in Sec.~\ref{sec:theory}.
We begin in Sec.~\ref{sec:ilq-nuc} by fitting the new I-Love-Q relations using the constrained set of EoSs.
This is followed in Sec.~\ref{sec:ilq-hyb} by an analysis and discussion into how well hybrid stars agree with the improved binary Love relations. 

\subsubsection{Nuclear matter stars}\label{sec:ilq-nuc}
Following the work of Ref.~\cite{Yagi:ILQ}, the data for each EoS-insensitive relation is first fit to the following curve:
\begin{equation}\label{eq:ILQfit}
\ln{y_i}=a_i+b_i \ln{x_i} + c_i (\ln{x_i})^2 + d_i (\ln{x_i})^3 + e_i (\ln{x_i})^4,
\end{equation}
where the updated coefficients are given in Table~\ref{tab:ILQfit}.
Additionally, we offer an improvement to the functional form of the I-Love-Q fitting curve. 
Namely, we use the Newtonian relationships between various observables as a controlling factor in the fit~\cite{Yagi:ILQ}:
\begin{equation}\label{eq:Newtonian}
\bar{I}^{\text{N}} = K_{\bar{I}\Lambda}\Lambda^{2/5}, \hspace{3mm} \bar{Q}^{\text{N}} = K_{\bar{Q}\Lambda}\Lambda^{1/5}, \hspace{3mm} \bar{I}^{\text{N}} = K_{\bar{I}\bar{Q}}\bar{Q}^{2}.
\end{equation}
This is appended to an expansion in $\Lambda^{-1/5} \propto C$, where $C=M/R$ is the compactness of the star.
This results in the following overall fitting relation:
\begin{equation}\label{eq:ILQfitNew}
y=K_{yx} x^{\alpha} \frac{1+\sum_{i=1}^3 a_i x^{-i/5}}{1+\sum_{i=1}^3 b_i x^{-i/5}},
\end{equation}
where $y$ and $x$ correspond to NS observables $\bar{I}$, $\bar{Q}$, and $\Lambda$, and $\alpha$ is given by $2/5$, $1/5$, and $2$ for the $\bar{I}-\Lambda$, $\bar{Q}-\Lambda$, and $\bar{I}-\bar{Q}$ relations, respectively.
The new fitting coefficients are presented in Table~\ref{tab:ILQfitNew}.
While the two above curves both result in fits with similar $R^2$ values (Coefficient of determination, defined as $R^2=\sum_i(f_i-\bar{y})^2/\sum_i(y_i-\bar{y})^2$, where $\bar{y}$ is the mean data value, and $f_i$, $y_i$ are the modeled and actual data values) of $\sim 0.9999995$ for the data presented, the latter has the advantage that it properly limits to the Newtonian case as $\Lambda \rightarrow \infty$~\cite{Yagi:binLove}.

\begin{table*}
\centering
\caption{
Updated fit parameters for the I-Love-Q relations, fitted to the constrained EoS data by the curve found in Eq.~\ref{eq:ILQfit}.
}\label{tab:ILQfit}
\begin{tabular}{ c  c | c c c c c } 
 \hline
 \hline
 $y_i$ & $x_i$ & $a_i$ & $b_i$ & $c_i$ & $d_i$ & $e_i$ \\
 \hline
 $\bar{I}$ & $\Lambda$ & $1.493$ & $0.06410$ & $0.02085$ & $-5.018 \times 10^{-4}$ & $3.16 \times 10^{-7}$ \\
 $\bar{Q}$ & $\Lambda$ & $0.2093$ & $0.07404$ & $-0.05382$ & $-5.018 \times 10^{-3}$ & $1.576 \times 10^{-4}$ \\ 
  $\bar{I}$ & $\bar{Q}$ & $1.383$ & $0.5931$ & $-0.02161$ & $0.04190$ & $-2.968 \times 10^{-3}$ \\
 \hline
 \hline
\end{tabular}
\end{table*}

\begin{table*}
\centering
\caption{
I-Love-Q and C-Love, relations fit parameters for the constrained EoS data using the improved fitting relations found in Eq.~\ref{eq:ILQfitNew}.
This fitting relation, unlike previous versions, properly limits to the Newtonian case as $\Lambda \rightarrow \infty$.
}\label{tab:ILQfitNew}
\begin{tabular}{ c  c  | c c c c c c c c} 
 \hline
 \hline
 $y$ & $x$ & $\alpha$ & $K_{yx}$ & $a_1$ & $a_2$ & $a_3$ & $b_1$ & $b_2$ & $b_3$ \\
 \hline
 $\bar{I}$ & $\Lambda$ & $2/5$ & $0.5313$ & $1.287$ & $0.09888$ & $-2.300$ & $-1.347$ & $0.3857$ & $-0.02870$\\
 $\bar{Q}$ & $\Lambda$ & $1/5$ & $3.555$ & $-2.122$ & $2.72$ & $-1.491$ & $0.8644$ & $-0.1428$ & $-1.397$\\
 $\bar{I}$ & $\bar{Q}$ & $2$ & $0.008921$ & $10.59$ & $-37.46$ & $43.18$ & $-2.361$ & $1.967$ & $-0.5678$\\
 $C$ & $\Lambda$ & $-1/5$ & $0.2496$ & $-919.6$ & $330.3$ & $-857.2$ & $-383.5$ & $192.5$ & $-811.1$\\
\hline
\hline
\end{tabular}
\end{table*}

Figure~\ref{fig:ILQ} shows the improved I-Love relations between the dimensionless moment of inertia $\bar{I} \equiv I/M^3$, the dimensionless quadrupole moment $\bar{Q} \equiv Q/M^3$, and the dimensionless tidal deformability $\Lambda$ for both the constrained and unconstrained sets of EoSs.
Here, each category of EoS are fit individually using the curve found in Eq.~\ref{eq:ILQfitNew}.
We observe that the constrained EoSs show considerable improvement from the unconstrained EoSs.
Additionally, we note a large improvement from the results of previous works~\cite{Yagi:ILQ}, validated in Tab.~\ref{tab:maxVar}, where the maximal EoS variation from each fit is tabulated; comparing the results of previous works to the unconstrained and constrained sets of EoSs.
Observe how the constrained EoSs outperform both other cases by a considerable amount for each I-Love-Q relation.

\begin{figure*}
\begin{center} 
\includegraphics[width=.32\textwidth]{IL.pdf}
\includegraphics[width=.32\textwidth]{QL.pdf}
\includegraphics[width=.32\textwidth]{IQ.pdf}
\end{center}
\caption{
Individual I-Love relations $\bar{I}-\Lambda$ (left), $\bar{Q}-\Lambda$ (center), and $\bar{I}-\bar{Q}$ (right), shown for both the constrained EoSs (solid green) and unconstrained EoSs (dotted maroon).
In these figures, the black dashed lines corresponds to the fits given by Eq.~\ref{eq:ILQfitNew} (note: each panel contains individual fits for the constrained and unconstrained EoSs, indistinguishable on such a large scale).
Observe how the fractional difference from the fits, shown in the bottom panels, is greatly suppressed for the constrained case, compared to both the unconstrained case, and results from previous works~\cite{Yagi:ILQ}.
The maximal EoS variation from the fits for the unconstrained and constrained sets of EoSs are compared in Tab.~\ref{tab:maxVar}.
Additionally shown in this figure is the fractional difference from the nuclear matter fits for the 10 hybrid star EoSs (dashed green).
}
\label{fig:ILQ}
\end{figure*} 


\begin{table}
\centering
\caption{
Comparison between the EoS-insensitive relations (I-Love-Q, C-Love, R-Love, and binary Love) maximal EoS variability for the results of previous works~\cite{Yagi:ILQ,Yagi:binLove}, and the unconstrained and constrained sets of EoSs analyzed here. 
The maximum EoS variation, given by the largest fractional difference (the absolute difference is also shown in parenthesis for the C-Love, R-Love, and binary Love fits) from the fits as shown in Figs.~\ref{fig:ILQ},~\ref{fig:clove},~\ref{fig:rlove} and~\ref{fig:binLove}, observes a considerable improvement for the constrained set of EoSs, compared to the unconstrained set as well as that found in previous works.
Additionally, note how the maximal EoS variation for the unconstrained set of EoSs is slightly larger than that found in Refs.~\cite{Yagi:ILQ,Yagi:binLove} - due to the effects of large random sampling taking into account more sources of uncertainty.
}\label{tab:maxVar}
\begin{tabular}{ c  || c c c } 
 \hline
 \hline
 \textbf{EoS-insensitive} & \multicolumn{3}{c}{\textbf{Maximal EoS Variability}} \\
 \cline{2-4}
 \textbf{Relation} & \multicolumn{1}{c|}{\emph{Previous}} & \multicolumn{1}{c|}{\emph{Unconstrained}} & \emph{Constrained}\\
 \hline
 $\bar{I}-\Lambda$ &  $0.0059$ & $0.007673$ & $0.003138$\\
 $\bar{Q}-\Lambda$ & $0.010$ & $0.01262$ & $0.004661$\\
 $\bar{I}-\bar{Q}$ & $0.012$ & $0.01490$ & $0.005716$\\
 \hline
 \noalign{\smallskip}
\noalign{\smallskip}
 \noalign{\smallskip}
\noalign{\smallskip}
\hline
 \hline
 \textbf{EoS-insensitive} & \multicolumn{3}{c}{\textbf{Maximal EoS Variability}} \\
 \cline{2-4}
 \textbf{Relation} & \multicolumn{1}{c|}{\emph{Previous}} & \multicolumn{1}{c|}{\emph{Unconstrained}} & \emph{Constrained}\\
 \hline
 \multirow{2}{*}{$C-\Lambda$} & $0.065$ & $0.07198$ & $0.02169$\\
 & (--) & ($0.01837$) & ($0.006582$)\\
  \hline
 \multirow{2}{*}{$R-\Lambda$} & -- & $0.02172$ & $0.05649$\\
 & (--) & ($358.9 \text{ m}$) & ($881.4 \text{ m}$)\\
 \hline
 $\Lambda_a-\Lambda_s$ & $\sim0.50$ & $0.5724$ & $0.2140$\\
 $q=0.90$ & (--) & ($190.1$) & ($37.23$) \\
 \cline{1-1}
 $\Lambda_a-\Lambda_s$ & $\sim0.20$ & $0.2447$ & $0.08284$\\
  $q=0.7$ & (--) & ($320.4$) & ($51.62$) \\
  \cline{1-1}
 $\Lambda_a-\Lambda_s$ & $\sim0.025$ & $0.03841$ & $0.01823$\\
  $q=0.50$ & (--) & ($244.1$) & ($28.96$) \\
  \cline{1-1}
\hline
\hline
\end{tabular}
\end{table}


These results indicate that EoS-insensitive relations can indeed be greatly improved upon by restricting to EoSs in agreeance with physical observations.
In addition, we confirm the validity of this method by noting that the unconstrained EoSs observe more EoS variability than that found in Ref.~\cite{Yagi:ILQ} due to the nature of large random sampling taking into account more sources of uncertainty than previously studied.
On the other hand, the constrained set of EoSs show significant improvement from that found in previous works.

\subsubsection{Hybrid quark-hadron stars}\label{sec:ilq-hyb}
In this section, we investigate the I-Love-Q universality of hybrid stars, and their compatibility with their nuclear matter counterparts.
We consider three different sets of data to be fit by Eq.~\ref{eq:ILQfitNew}:
\begin{enumerate}
\item Complete set of 100 constrained EoSs combined with the 10 hybrid star EoSs,
\item Complete set of 100 constrained EoSs alone,
\item Complete set of 10 hybrid star EoSs alone.
\end{enumerate}
Following this, we compute the fractional difference from the fits for all three cases for the 10 hybrid star EoSs.
The fractional differences for the second case (fit to only the constrained EoSs) is shown by the dashed green lines in Fig.~\ref{fig:ILQ} for example.
To compare the three different fits, Tab.~\ref{tab:hybridCompare} displays the maximal EoS variation in the $I-\Lambda$ relation for both the constrained and hybrid star EoSs in each case.

\begin{table}
\centering
\caption{
Comparison between the $I-\Lambda$ maximal EoS variability (fractional difference from the fit) for the constrained EoSs and the hybrid EoSs for three different cases of fitting data to Eq.~\ref{eq:ILQfitNew}: (1) fit to the combined data of constrained and hybrid EoS, (2) fit to the constrained EoS data alone, and (3) fit to the hybrid EoS data alone.
Observe how for all 3 cases, the hybrid EoSs are only universal up to a minimum of $\sim1$\%, while in each case the constrained EoSs outperform the hybrid ones.
The second case is demonstrated in Fig.~\ref{fig:ILQ}.
}\label{tab:hybridCompare}
\begin{tabular}{ c  || c c } 
 \hline
 \hline
 \textbf{Fitting} & \multicolumn{2}{c}{\textbf{Maximal EoS Variability}} \\
 \cline{2-3}
 \textbf{Case} &  \multicolumn{1}{c|}{\emph{Constrained}} & \emph{Hybrid}\\
 \hline
 \emph{Combined} &  \multirow{2}{*}{$0.004407$} & \multirow{2}{*}{$0.01363$}\\
 \emph{(Case 1)} & &\\
 \cline{1-1}
 \emph{Constrained only} & \multirow{2}{*}{$0.003138$} & \multirow{2}{*}{$0.01736$}\\
  \emph{(Case 2)} & &\\
  \cline{1-1}
 \emph{Hybrid only} & \multirow{2}{*}{$0.008400$} & \multirow{2}{*}{$0.01017$}\\
  \emph{(Case 3)} & &\\
  \cline{1-1}
\hline
\hline
\end{tabular}
\end{table}

Observe how the hybrid star EoSs obey the I-Love-Q relations in each case by up to $\sim1.7$\%, slightly higher than that found for nuclear matter EoSs in previous works~\cite{Yagi:ILQ}.
Further, we observe that the universality can not be improved by much through the introduction of new fits, only bringing the max EoS variation down to $\sim1$\% for the fits constructed with only hybrid star EoSs.
Concluding, we claim that hybrid star EoSs \emph{do} obey the I-Love-Q relations computed through nuclear EoS data, with a decrease in universality to $\sim1.7$\%.  

\subsection{C-Love Relations}\label{sec:clove}
In this section we consider approximately EoS-insensitive C-Love relations between NS compactness $C \equiv M/R$ and unitless tidal deformability $\Lambda$, as discussed by Yagi and Yunes~\cite{Yagi:binLove}.
Additionally in this section, we compute the relationship between the NS radius and tidal deformability from the C-Love relations.
This allows one to compare the approximate uncertainty on NS radius computed from the C-Love relations for both the constrained and unconstrained sets of EoSs. 
Similar to the previous section, we consider two classes of stars in this section: nuclear matter EoSs and hybrid quark-hadron star EoSs.

\subsubsection{Nuclear matter stars}\label{sec:clove-nuc}
Following Ref.~\cite{Yagi:binLove}, we begin by fitting the data for each set of EoSs to the simple curve:
\begin{equation}
C = \sum^2_{k=0} a_k (\ln{\Lambda})^k.
\end{equation}
Doing so yields $a_0 = 0.3617$, $a_1 = -0.03548$, and $a_2 = 0.0006194$, similar to that found in Ref.~\cite{Yagi:binLove}.
Additionally we offer an improvement to the fits as was done in Sec.~\ref{sec:ilq}, by considering the fitting function found in Eq.~\ref{eq:ILQfitNew}, with the controlling factor given by:
\begin{equation}
C^N=K_{C\Lambda}\Lambda^{-1/5}.\label{eq:cloveFit}
\end{equation}
The new fitting coefficients are displayed in Tab.~\ref{tab:ILQfitNew}, alongside the previously found I-Love-Q relations.

Figure~\ref{fig:clove} presents the C-Love relations for both the constrained and unconstrained sets of EoSs, along with the relative fractional difference from the fits. 
Observe how the constrained EoSs show a large suppression in EoS variability compared to the unconstrained EoSs, as well as to that found in previous works~\cite{Yagi:binLove}.
The maximal EoS variation is compared between the above three cases in Tab.~\ref{tab:maxVar}.
\begin{figure}
\begin{center} 
\includegraphics[width=\columnwidth]{CL.eps}
\end{center}
\caption{
Similar to Fig.~\ref{fig:ILQ} but for C-Love relations.
In this figure, we similarly construct fits to  Eq.~\ref{eq:ILQfitNew}, properly limiting to the Newtonian case $C \sim \Lambda^{-1/5}$.
Observe how, as before, the fractional difference from the fits is greatly suppressed for the constrained case, compared to both the unconstrained case, and results from previous works~\cite{Yagi:binLove}.
The maximal EoS variation from the fits for the unconstrained and constrained sets of EoSs are compared in Tab.~\ref{tab:maxVar}.
Additionally shown in this figure is the hybrid star C-Love relations along with their fractional difference from the constrained EoS fits (dashed green).
}
\label{fig:clove}
\end{figure} 


In addition, Fig.~\ref{fig:rlove} displays the relationship between the NS radius and tidal deformability for the constrained and unconstrained sets of EoSs.
This relationship is computed directly from the C-Love relations discussed above as $R(\Lambda)=m/C(\Lambda)$ for observed mass $m=1.48\text{ M}_{\odot}$ from GW170817.
We also present the uncertainty on a NS radius measurement using the C-Love relations in the bottom panel of Fig.~\ref{fig:rlove}, defined as the absolute difference from the fits.
This indicates that one can reliably measure the NS radius within $\sim 180 \text{ m} -350$ m, using the C-Love relations computed using the constrained set of EoSs.
We further observe a large suppression in the radius measurement accuracy from the unconstrained set of EoSs to the constrained set, tabulated in Tab.~\ref{tab:maxVar} as the maximal EoS measurement uncertainty.

\begin{figure}
\begin{center} 
\includegraphics[width=\columnwidth]{RL.eps}
\end{center}
\caption{
Similar to Fig.~\ref{fig:clove} but for the R-Love relationship between the NS radius and tidal deformability.
This relationship is computed directly from the C-Love relations, and estimates the uncertainty on the NS radius as computed through the C-Love relations.
Observe how the absolute difference from the fit is reduced significantly between the unconstrained and constrained sets of EoSs, shown by the comparisons in Tab.~\ref{tab:maxVar}.
}
\label{fig:rlove}
\end{figure} 

The results for the C-Love and R-Love relations, similar to that found in Sec.~\ref{sec:ilq}, show the considerable improvement found in using constrained EoSs over any unconstrained set.
We also note that the unconstrained EoSs display slightly larger EoS variability than found in Ref.~\cite{Yagi:binLove}, due to the large random sampling as discussed previously in Sec.~\ref{sec:ilq}.


\subsubsection{Hybrid quark-hadron stars}\label{sec:clove-hyb}
In this section, we consider the effect of hybrid star EoSs on the C-Love relation.
Much like Sec.~\ref{sec:ilq-hyb}, we perform 3 separate fits (constrained and hybrid EoSs combined, constrained EoSs only, and hybrid EoSs only) and compare the maximal EoS variation for each class of EoSs.

Table~\ref{tab:hybridCompareClove} compares the maximal EoS variability for the constrained and hybrid star EoSs under the fitting conditions for each case described above. 
We observe very similar behavior to that described in Sec.~\ref{sec:ilq-hyb}, in which the maximal EoS variation for hybrid stars only fluctuate slightly ($\sim 4.5\% - 7\%$) for each case.
From this, we similarly conclude that hybrid stars do obey the C-Love relation derived with nuclear matter stars, with the caveat that the maximum universality increases to $\sim 7.1\%$.
This is exemplified in Fig.~\ref{fig:clove} by the dashed green curves.

\begin{table}
\centering
\caption{
Similar to Tab.~\ref{tab:hybridCompare} but for the C-Love relation.
Observe how for all 3 cases, the hybrid EoSs are only universal up to a minimum of $\sim1$\% (fractional difference from the fit), while in each case the constrained EoSs outperform the hybrid ones.
The second case is demonstrated in Fig.~\ref{fig:clove}.
}\label{tab:hybridCompareClove}
\begin{tabular}{ c  || c c } 
 \hline
 \hline
 \textbf{Fitting} & \multicolumn{2}{c}{\textbf{Maximal EoS Variability}} \\
 \cline{2-3}
 \textbf{Case} &  \multicolumn{1}{c|}{\emph{Constrained}} & \emph{Hybrid}\\
 \hline
 \emph{Combined} &  \multirow{2}{*}{$0.03686$} & \multirow{2}{*}{$0.05533$}\\
 \emph{(Case 1)} & &\\
 \cline{1-1}
 \emph{Constrained only} & \multirow{2}{*}{$0.02169$} & \multirow{2}{*}{$0.07191$}\\
  \emph{(Case 2)} & &\\
  \cline{1-1}
 \emph{Hybrid only} & \multirow{2}{*}{$0.05788$} & \multirow{2}{*}{$0.04478$}\\
  \emph{(Case 3)} & &\\
  \cline{1-1}
\hline
\hline
\end{tabular}
\end{table}

\subsection{Binary love relations}\label{sec:binary}
Next we consider improvements to the binary Love relations.
Similar to Sec.~\ref{sec:ilq}, we consider two distinct classes of NSs: nuclear matter EoSs and hybrid quark-hadron star EoSs~\cite{Paschalidis2018,Alford:2017qgh,1971SvA....15..347S,Zdunik:2012dj,Alford:2013aca}.
Following Ref.~\cite{Yagi:binLove}, we begin by fitting the binary Love relations to the constrained and unconstrained sets of EoSs using the two-dimensional curve:
\begin{equation}\label{eq:binLovefit}
\Lambda_a=F_n(q) \frac{1+ \sum_{i=1}^3 \sum_{j=1}^2 b_{ij}q^j x^{i/5}}{1 + \sum_{i=1}^3 \sum_{j=1}^2 c_{ij}q^j x^{i/5}} \Lambda_s^{\alpha}.
\end{equation}
Here, q is the mass ratio $q \equiv m_1/m_2$ with $m_1 \leq m_2$, and $F_n(q)$ is the Newtonian-limiting control factor given by:
\begin{equation}\label{eq:control}
F_n(q) \equiv \frac{1-q^{10/(3-n)}}{1+q^{10/(3-n)}}.
\end{equation}
The updated fit parameters for the constrained set of EoSs can be found in Table~\ref{tab:binLovefit}.
Notice how, unlike the individual NS I-Love-Q relations, these relations also depend on the ratio of constituent masses in the binary system.
\begin{table*}
\centering
\caption{
Updated fit parameters for the binary Love relations, as given by the curve found in Eq.~\ref{eq:binLovefit}.
}\label{tab:binLovefit}
\addtolength{\tabcolsep}{1pt} 
\begin{tabular}{c  c  c  c  c  c  c  c c c c c c c} 
 \hline
 \hline
$n$ & $\alpha$ & $b_{11}$ & $b_{12}$ & $b_{21}$ & $b_{22}$ & $b_{31}$ & $b_{32}$ & $c_{11}$ & $c_{12}$ & $c_{21}$ & $c_{22}$ & $c_{31}$ & $c_{32}$\\
\hline
$0.743$ & $-1$ & $-14.40$ & $14.45$ & $31.36$ & $-32.25$ & $-22.44$ & $20.35$ & $-15.25$ & $15.37$ & $37.33$ & $-43.20$ & $-29.93$ & $35.18$\\
 \hline
 \hline
\end{tabular}
\addtolength{\tabcolsep}{-1pt}
\end{table*}

Figure~\ref{fig:binLove} shows the improved binary Love relation for 3 different mass ratios: $q=0.9$, $q=0.75$, and $q=0.5$ for both the constrained and unconstrained sets of EoSs.
Observe how once again, the constrained set of EoSs show a considerable improvement upon both the unconstrained set of EoSs, as well as the results found in previous works~\cite{Yagi:binLove}.
Additionally, the unconstrained set of EoSs show similar, yet slightly larger EoS variation than was previously found in Ref.~\cite{Yagi:binLove} due to the random sampling taken into account in this analysis.
Once again, the maximum EoS variability (in terms of both absolute and fractional differences from the fit) is tabulated in Tab.~\ref{tab:maxVar} for each value of mass ratio considered here.
We take note that while the fractional errors increase rapidly with increasing mass ratio $q$, the absolute errors remain consistent, with a subtle maximum at $q=0.75$. 
This can be explained by the control factor $F_n(q)$ given by Eq.~\ref{eq:control}, which similarly observes a peak for central mass ratio values of $q\approx0.75$.
We additionally make two observations about the effective ranges of $\Lambda_s$ seen in Fig.~\ref{fig:binLove}: (i) as the constituent masses in the binary system become further separated, the range of $\Lambda_s$ values shrinks; and (ii) the constrained set of EoSs typically display smaller values of $\Lambda_s$ than the unconstrained set do.
The former observation can be explained by the following reasoning: as the masses further separate, there is a limited range of tidal deformabilities $\Lambda_{1,2}$ in the M-L relationships which satisfy the mass ratio constraint.
From this we suggest the reader take note that as the mass ratio decreases, the allowable paramreter space for the binary Love relations shrinks.
The latter observation originates from the EoS generation process: the unconstrained EoSs are randomly generated -- resulting in various EoSs which are both stiffer and softer than those constrained by GW170817.
This results in the unconstrained EoSs containing a larger range of tidal deformabilities $\Lambda$ (both above and below) than the constrained ones, which ultimately gives \emph{larger} values of $\Lambda_s=\frac{1}{2}(\Lambda_1+\Lambda_2)$.

\begin{figure}
\begin{center} 
\includegraphics[width=\linewidth]{binLoveAbsolute.eps}%{binLove.eps}
\end{center}
\caption{
Binary Love relations shown for the constrained EoSs (dotted maroon) and unconstrained EoSs (solid green) for various values of mass ratio: $q=0.9$, $q=0.75$, and $q=0.50$.
In this figure, the top panel displays the EoS-insensitive relations with fits given by Eq.~\ref{eq:binLovefit} shown by dashed black lines (note there are 2 fits for each mass ratio, corresponding to the constrained and unconstrained sets of EoSs), while the bottom 3 panels correspond to the absolute EoS variation for each mass ratio.
Observe how the constrained set of EoSs show a reduction in EoS variation for both the unconstrained set, and from the results shown in Ref.~\cite{Yagi:binLove}.
The maximal EoS variability (shown as a the fractional difference for comparison purposes) for each case is tabulated and compared in Tab.~\ref{tab:maxVar}.
Additionally shown are the binary Love relations for the 10 hybrid star EoSs (dashed bright green curves).
Observe the large deviations from the fit as the mass ratio $q$ increases, returning near the fit before the transitional pressure at large values of $\Lambda_s$.
}
\label{fig:binLove}
\end{figure} 

Additionally shown in Fig.~\ref{fig:binLove} are the binary Love relations for the 10 hybrid EoSs discussed in previous sections.
As described in Sec.~\ref{sec:theory}, transitional quark-hadron matter stars undergo first-order phase transitions at pressure $P_{\text{tr}}$, where the hadronic branch departs into a quark-matter branch at the corresponding transitional mass $M_{\text{tr}}$.
These transitions result in two distinct types of NSs, based on their observed mass: (i) massive ($M \geq M_{\text{tr}}$) hybrid stars which have quark-matter cores and nuclear matter crusts (we denote these as ``HS"), and (ii) low-mass ($M \leq M_{\text{tr}}$) hadronic stars with no internal transition to quark matter (we denote these as ``NS").
Contrary to the single-star relations (I-Love-Q, C-Love, and R-Love) where the hybrid stars moderately agree with their nuclear matter counterparts, the binary-star relations show large disagreement, which can be seen in Fig.~\ref{fig:binLove}.
Observe how the agreement with the fit is quite good at low pressures (large $\Lambda_s$) where the the binaries consist of NS/NS combinations (here, both stars' EoSs consist of the same class of nuclear matter as the constrained EoSs).
Once the critical pressure is reached, one or both stars transition into the hybrid branches, resulting in large reductions in tidal deformability as shown in Fig.~\ref{fig:hybridML}.
When one star lies on the hadronic-matter branch while the other is on the quark-matter branch, 
the difference between tidal deformabilities becomes larger than expected for pure hadronic-matter stars.
This results in a large deviation in the sums and differences of tidal deformabilities $(\Lambda_1 \pm \Lambda_2)$, disrupting the overall universality and generating the large ``bump" in the binary love relations for hybrid stars\footnote{This discrepancy is not present in the I-Love-Q and C-Love relations due to their singular nature.} seen in Fig~\ref{fig:binLove}.

Due to the inconsistencies in the sums and differences of tidal deformabilities for binary hybrid star systems, we conclude that hybrid stars do \emph{not} agree with the traditional nuclear matter binary Love relations.
A number of attempts were made to reconcile these differences, such as performing separate ``HS" binary Love relations fitted directly to the hybrid star branches of the EoSs.
This analysis quickly left the scope of the paper due to the large number of parameters required to control the varying transitional pressures through different EoSs.
Therefore, we leave this analysis on the hybrid star binary Love relations to be discussed in future works.
There, we anticipate generating large numbers of hybrid star EoSs with varying transition masses to be taken into account as an additional fitting parameter in a piece-wise fitting function designed to take into account the ``bumps" in the binary Love relations.

\iffalse%%%%%%%%%%%%%%%%%%% COMMENTED OUT %%%%%%%%%%%%%%%%%%%%%%%%%%%%%%%%%%%
\subsubsection{Hybrid quark-hadron stars}\label{sec:binLove-hybrid}
As described in Sec.~\ref{sec:theory}, transitional quark-hadron matter stars undergo first-order phase transitions at pressure $P_{\text{tr}}$, where the hadronic branch departs into a quark-matter branch at the corresponding transitional mass $M_{\text{tr}}$.
These transitions result in two distinct types of NSs, based on their observed mass: (i) massive ($M \geq M_{\text{tr}}$) hybrid stars which have quark-matter cores and nuclear matter crusts (we denote this as ``HS"), and (ii) low-mass ($M \leq M_{\text{tr}}$) hadronic stars with no internal transition to quark matter (we denote this as ``NS").
We consider 3 classes of binary systems in this analysis:
\begin{itemize}
\item HS/NS binary: consists of one massive hybrid star, and one small hadronic neutron star,
\item NS/NS binary: consists of two small-mass hadronic neutron stars,
\item HS/HS binary: consists of two massive hybrid stars.
\end{itemize}

To determine how each of class of binary NS systems obey our improved binary Love relations, we pick a representative set of 10 binary pairs from each class from the ACB5 hybrid EoS, with a transition at $M_{\text{tr}} \approx 1.4 \text{ M}_{\odot}$.
Binary Love values are computed for each pair, and fractional residuals from the binary Love curve found in Sec.~\ref{sec:binary} are determined and compared to the values from the constrained set of EoSs.
The results are shown in Fig.~\ref{fig:hybrid}.
First, we note that the NS/NS combination of stars gives consistent results to that found earlier -- as expected.
Next, we observe that HS/HS combinations agree moderately with previous relations\footnote{Note that these EoSs do not fit within GW170817s 90\% posterior on EoS - so we expect deviations consistent with the ``unconstrained" set of EoSs, rather than the ``constrained" set.}, while HS/NS combinations disagree significantly.
This can be explained by the large reduction in tidal deformability proceeding the transition pressure for hybrid stars, as shown in Fig.~\ref{fig:hybrid}.
When one star lies on the hadronic-matter branch while the other is on the quark-matter branch, 
the difference between tidal deformabilities becomes larger than expected for pure hadronic-matter stars.
This results in a large deviation in the NS binary Love relations $\frac{1}{2}(\Lambda_1 \pm \Lambda_2)$, disrupting the overall universality.

\begin{figure}
\begin{center} 
\includegraphics[width=\columnwidth]{hybrid.eps}
\end{center}
\caption{
Fractional difference from the improved binary Love relations fit for a representative set of constrained nuclear matter EoSs (gray circles), and 30 selected NS binaries from the ACB5 hybrid EoS.
In particular, these binary pairs are selected from each of the following categories: (i) binaries with which one massive star has a quark matter core, while the other is pure hadronic matter (i.e. small enough mass that the transition point has not yet been reached), denoted HS/NS (green triangles); (ii) binaries with two small-mass hadron stars, denoted NS/NS (red squares); and (iii) binaries with two large-mass stars with quark matter cores, denoted HS/HS (magenta diamonds).
}
\label{fig:hybrid}
\end{figure} 

This discrepancy between the hybrid and hadronic matter EoSs can be seen in Fig.~\ref{fig:binLove}, where the binary Love relations for the 10 hybrid star EoSs are shown alongside the constrained and unconstrained nuclear matter EoSs.
Observe how the residuals from the fit are quite good at low pressures (large $\Lambda_s$) where the the binaries consist of NS/NS combinations, before reaching the transitional pressure where the deviations become increasingly large ($\sim160$\% for $q=0.90$, $\sim37$\% for $q=0.75$, and $\sim6$\% for $q=0.50$) for HS/HS and HS/NS binaries.

To further investigate this phenomena, we consider binary Love fits to the HS or NS branches \emph{individually}.
It was found that the binary Love relations failed to properly fit to the collection of full hybrid star EoSs, due to the varying transitional masses $M_{\text{tr}}$ across the various EoSs.
To compensate for these differences, we separate the two branches into the NS (hadronic matter) and HS (hybrid matter) branches.
We first observe that the NS branches obey the ``hadronic" binary Love relations derived in Sec.~\ref{sec:binary} within the same EoS variability, indicating no further need for additional fits.
However, the HS branches do not follow the same relations, suggesting the need for a new ``hybrid binary love fit" pertaining to this branch alone.
We begin by fitting the binary Love curve in Eq.~\ref{eq:binLovefit} to the HS branches, resulting in the fit coefficients tabulated in the bottom row of Tab.~\ref{tab:binLovefit}.
We note here that the Newtonian limit for this fit is not physical for quark-star matter, which does not reach low enough central densities.
Attempts were made to remedy this by allowing the polytropic index $n$ to be either $0$ or a free fitting parameter, both of which failed.
We instead leave the fits as they are, with the disclaimer that they do not properly limit to the Newtonian case.
The fits are displayed in the top panel of Fig.~\ref{fig:binLove} as dot-dashed blue curves, showing a good relationship with the hybrid star branches used in the analysis.
We urge caution with the use of these fits, as the limited sample size of 10 EoSs results in a less accurate fit than its nuclear-matter counterpart.
Similar to the previous section, the fractional difference from the fit for each HS branch is computed, resulting in a maximum EoS variability of $18.49$\% for $q=0.90$.
This result is strongly consistent with the results found for the nuclear matter fits of Sec.~\ref{sec:binary}.

We conclude this section with the observation that the hybrid star EoSs do \emph{not} conform to the fits performed on nuclear matter EoSs.
This is not surprising, as the hybrid branch of the EoSs forces inconsistent results between $\Lambda_1$ and $\Lambda_2$ with different masses; an issue not present for the single-star I-Love-Q relations.
By separating the two branches of the hybrid EoSs, we find a new ``hybrid" branch binary Love fit which gives consistent results to that found for the nuclear matter fits. 
\fi%%%%%%%%%%%%%%%%%%% COMMENTED OUT %%%%%%%%%%%%%%%%%%%%%%%%%%%%%%%%%%%

%%%%%%%%%%%%%%%%%%%%%%%%% Begin Future Observations %%%%%%%%%%%%%%%%%%%%%%%%%%%%%%%%%%%%%%%%%%%%%%%%%%%%%%%%%%%%%%%%%%%%%%%%%%%%%%%%%%%%%%%%%%%%%%%%%%%%%%%%%%%%%%%%%%%%%%%%%%%%%%%%%%%%%%%%%%%%%%%%%%%%%
\section{Impact on future observations}\label{sec:observations}

We have now shown that binary NS merger observations can help improve EoS-insensitive relations - the question is: is it worth it?
Current interferometer sensitivities are not yet small enough to accurately constrain $\tilde{\Lambda}$.
For example, GW170817 was detected by LIGO observing run 2 (``O2")~\cite{aLIGO} and was able to constrain $\tilde{\Lambda}$ to a $90\%$ confidence interval of 325 (or $\sigma_{\tilde{\Lambda}}=198$) centered at $\mu_{\tilde{\Lambda}}=395$.
This corresponds to statistical uncertainties on the order of $\sim 82\%$, which vastly dominates the error budget compared to the small systematic errors picked up by EoS variation in the EoS-insensitive relations.
This implies that currently, the use of improved EoS-insensitive relations will only make a negligible difference on the extraction of tidal parameters.

In Sec.~\ref{sec:marginalization}, we first compute the systematic errors introduced by using the improved binary Love relations derived earlier.
In Sec.~\ref{sec:futureObservations}, we compare the statistical errors on the parameter extraction of tidal parameters to the above-mentioned systematic errors.
This is repeated for 5 future detectors, where multiple event detections become important.
Because future events occur with weighted tidal parameters $\tilde\Lambda$ which are unknown at this time, we can not accurately combine the uncertainties on $\tilde\Lambda$ for multiple events.
To remedy this, we re-parameterize the waveform and instead consider the $\lambda_0$ and $\lambda_1$ tidal coefficients produced by Taylor expanding the dimensionless tidal deformability $\Lambda$ about the ``canonical" reference mass of $m_0=1.4\text{ M}_{\odot}$~\cite{delPozzo:TaylorTidal,Yagi:binLove}:
\begin{equation}
\lambda \approx \lambda_0 + \lambda_1 (1-\frac{m}{m_0}),
\end{equation}
where $\lambda_0$ and $\lambda_1$ are the tidal deformability and its slope at $1.4 \text{ M}_{\odot}$\footnote{$\lambda_0$ is commonly known as $\Lambda_{1.4}$.}. 
The new tidal parameters $\lambda_0$ and $\lambda_1$ do not depend on mass, thus are identical for every future binary NS merger, and may be combined in uncertainty.
Under this reasoning, we consider the new tidal parameters $\lambda_0$ and $\lambda_1$ for the remainder of the investigation.

\subsection{Error Marginalization}\label{sec:marginalization}

In this section, we investigate the residuals on $\Lambda_a(\Lambda_s,q)$ in order to marginalize over the intrinsic error found in the binary Love EoS-insensitive relations.
In particular, we restrict our focus to only the constrained set of EoSs.
Residuals in $\Lambda_a$ are computed as $\Lambda_a^{\text{relation}}(\Lambda_s,q)-\Lambda_a^{\text{true}}$, where $\Lambda_a^{\text{true}}$ corresponds to the true value predicted by the sample EoSs, and $\Lambda_a^{\text{relation}}(\Lambda_s,q)$ corresponds to the value obtained through the EoS-insensitive relations found in Sec.~\ref{sec:binary}.
We follow this up by converting the $\Lambda_a$ residuals into $\lambda_0$, allowing us to approximate the overall systematic errors on $\lambda_0$ from using the improved binary Love relations.

Following Ref.~\cite{Katerina:residuals}, we assume the residuals on $\Lambda_a$ observe a Gaussian distribution with mean and standard deviation given by:
\begin{align}
\mu_{\Lambda_a}(\Lambda_s,q) &=\frac{\mu_{\Lambda_s}(\Lambda_s)+\mu_{q}(q)}{2},\\ 
\sigma_{\Lambda_a} &=\sqrt{\sigma_{\Lambda_s}^2(\Lambda_s) + \sigma_{q}^2(q)}. 
\end{align}
Similar to Ref.~\cite{Katerina:residuals}, we fit the individual components to the following power functions:
\begin{align}
\mu_{\Lambda_s}(x) &= \mu_1 x + \mu_2, \label{eq:margFit1}\\ 
\mu_{q}(x) &= \mu_3 x^2 + \mu_4 x + \mu_5, \label{eq:margFit2}\\ 
\sigma_{\Lambda_s}(x) &= \sigma_1 x^{5/2} + \sigma_2 x^{3/2} + \sigma_3 x \sigma_4 x^{1/2} + \sigma_5, \label{eq:margFit3}\\ 
\sigma_{q}(x) &= \sigma_6 x^3 + \sigma_7 x^2 + \sigma_8 x + \sigma_9. \label{eq:margFit4}
\end{align}
The fitting parameters $\mu_i$ and $\sigma_i$ are tabulated in Table~\ref{tab:marginalized}.

\begin{table}
\centering
\caption{
Coefficients to the fits given by Eqs.(~\ref{eq:margFit1})-(\ref{eq:margFit4}) for the relative error on $\Lambda_a$ in the improved binary Love EoS-insensitive relations presented in this paper.
}\label{tab:marginalized}
\addtolength{\tabcolsep}{1pt} 
\begin{tabular}{ c | c || c | c}
\hline 
\noalign{\smallskip}
$\mu_1$ & $3.509 \times 10^{-3}$ & $\sigma_1$ & $-2.074 \times 10^{-7}$\\
$\mu_2$ & $9.351 \times 10^{-1}$ & $\sigma_2$ & $-1.492 \times 10^{-3}$\\
$\mu_3$ & $-18.07$ & $\sigma_3$ & $-4.891 \times 10^{-2}$\\
$\mu_4$ & $27.56$ & $\sigma_4$ & $8.207 \times 10^{-1}$\\
$\mu_5$ & $-10.10$ & $\sigma_5$ & $-1.308$\\
 &  & $\sigma_6$ & $-63.76$\\
 &  & $\sigma_7$ & $11.14$\\
 &  & $\sigma_8$ & $75.25$\\
 &  & $\sigma_8$ & $-23.69$\\
 \noalign{\smallskip}
 \hline
\end{tabular}
\addtolength{\tabcolsep}{-1pt}
\end{table}

Now that we have obtained the residuals on $\Lambda_a$, we must convert them into the new tidal parameter $\lambda_0$, in order to compute and easily compare the systematic errors to the statistical errors on $\lambda_0$, found in Sec.~\ref{sec:futureObservations}.
Figure~\ref{fig:qLsResiduals} shows the standard deviations of these residuals (converted into $\lambda_0$) binned in both $q$ and $\Lambda_s$ as was done above. 
To estimate the overall systematic errors on $\lambda_0$ accrued upon the use of the improved binary Love relations, we combine the errors in both $q$ and $\Lambda_s$.
Figure~\ref{fig:residuals} displays the un-binned Gaussian distribution of $\lambda_0$ residuals for both the constrained, and unconstrained sets of EoSs for comparison.
Observe how the standard deviations $\sigma=9.764$ and $\sigma=78.28$ show a large decrease from the unconstrained to the constrained sets of EoSs.
Additionally, we find the 90th, 99th, and 100th percentiles on $\lambda_0$ to be $P_{90}=13.19$, $P_{99}=42.38$, and $P_{100}=111.32$ for the constrained EoSs.
Because the un-binned residuals seen in Fig.~\ref{fig:residuals} are dominated by the low-error regions of the parameter space ($\Lambda_s \rightarrow 0$, $\Lambda_s \rightarrow \infty$, $q \rightarrow 0$ and $q \rightarrow 1$) shown by Fig.~\ref{fig:qLsResiduals}, we consider the 90th percentile error (indicated by the dashed green line in Fig.~\ref{fig:qLsResiduals}) for the remainder of the analysis, in order to further take into account the high-error regions of the parameter space ($\Lambda_s \sim 2000$, $q\sim0.5$).
We take this value of $P_{90}=13.19$ to be the systematic measurement error introduced by using the improved binary Love relations on the parameter extraction of $\lambda_0$, indicated by the dashed indigo line in Fig.~\ref{fig:stackedFisher}.
Similarly, we derive the systematic measurement error on $\tilde\Lambda$ to be $12.01$, shown in Fig.~\ref{fig:singleFisherLt}.
Currently, the statistical error of $170.1$ on $\lambda_0$ from GW170817 dominates the error budget compared to the systematic error of $13.19$, as seen in Fig.~\ref{fig:stackedFisher}.
This means that, for the current ``O2" detector sensitivity, the use of improved binary Love relations will only make a negligible difference on the extraction of $\tilde\Lambda$.
However, future detectors (for example aLIGO, A\texttt{+}, Voyager, CE, and ET) are planned with large reductions in sensitivity - both decreasing the statistical errors and allowing for a larger binary NS merger detection rate; further reducing the statistical uncertainties.
In Sec.~\ref{sec:observations}, we analyze this further and discuss when the systematic errors from EoS-insensitive relations become comparable to the statistical errors.

\begin{figure}
\begin{center} 
\includegraphics[width=\columnwidth]{qResiduals.eps}
\includegraphics[width=\columnwidth]{LsResiduals.eps}
\end{center}
\caption{$\lambda_0$ residuals binned in $q$ (top) and $\Lambda_s$ (bottom), highlighting the different error weights across the entire $(q,\Lambda_s)$ parameter space.
In this figure, the violet circles indicate the standard deviation of each bin in $q$-space ($\Lambda_s$-space), while the black dashed lines represent the best fit given by Eqs. (\ref{eq:margFit3})-(\ref{eq:margFit4}).
Observe how the error is maximal for $q\sim0.5$ and $\Lambda_s\sim2000$, while it becomes minimal for both low and high values of $q$ and $\Lambda_s$.
Also shown in the figure is the 90th percentile of the un-binned residuals seen in Fig.~\ref{fig:residuals}, taken to be the overall systematic uncertainty introduced by using binary Love relations.
}
\label{fig:qLsResiduals}
\end{figure}

\begin{figure}
\begin{center} 
\includegraphics[width=\columnwidth]{residuals.pdf}
\end{center}
\caption{
Residuals on $\lambda_0$ computed as $\lambda_0^{\text{fit}}-\lambda_0^{\text{true}}$ for the binary Love relations modeled in Sec.~\ref{sec:binary} for both constrained and unconstrained sets of EoSs.
These residuals obey Gaussian distributions centered at $\mu=-0.1530$ and $\mu=0.2710$ with standard deviations of $\sigma=9.764$ and $\sigma=78.28$ for the constrained and unconstrained sets of EoSs, respectively.
These uncertainties correspond roughly to the systematic errors introduced on the parameter extraction of $\lambda_0$ upon the use of binary Love relations.
However, to take into account the systematic errors found in high-error regions of the parameter space, we instead set the systematic error to be the 90th percentile, $P_{90}=13.19$.
Observe that the systematic errors from using the improved (constrained) binary Love relations are negligible compared to the statistical errors accrued on parameter extraction from GW170817, found to be $\sigma_{\lambda_0}=170.1$.
}
\label{fig:residuals}
\end{figure}

\subsection{Future Observations}\label{sec:futureObservations}
In this section, we estimate the feasibility of utilizing the improved EoS-insensitive relations in future binary NS merger events.
This estimate is acquired through a simple Fisher analysis~\cite{Finn:Fisher,Cutler:Fisher} which approximates the accuracy with which one can extract best-fit parameters $\theta^a$, given a prior template waveform.
For the remainder of the paper, we consider a template parameter vector consisting of:
\begin{equation}\label{eq:template}
\theta^a=(\ln{A},\phi_c,t_c,\ln{\mathcal{M}},\ln{\mathcal{\eta}},\chi_s,\chi_a,\lambda_0, \lambda_1),
\end{equation}
where $A \equiv \sqrt{\frac{2 \eta}{3 \pi^{1/3}}}$ is a normalized amplitude factor, $\eta \equiv m_1 m_2/M^2$ is the symmetric mass ratio with $m_{1,2}$ and $M$ being the individual and total masses, $\mathcal{M}=M \eta^{3/5}$ is the chirp mass, and $\chi_{s,a}=\frac{1}{2}(\chi_1\pm\chi_2)$ are the symmetric and antisymmetric total spins, where $\chi_{1,2}$ are the NSs individual spins. 
Following Refs.~\cite{Cutler:Fisher,Berti:Fisher,Poisson:Fisher}, this method relies on the crude assumption of Gaussian prior distributions\footnote{Typically, the more valid assumption is a uniform prior distribution; an improvement made in a more detailed Bayesian analysis.}.
The resulting posterior distribution is Gaussian with root-mean-square given by:
\begin{equation}
\Delta \theta^a=\sqrt{\Big( \tilde{\Gamma}^{-1}\Big)^{aa}}.
\end{equation}
Here, the Fisher matrix $\tilde{\Gamma}$ is defined by:
\begin{equation}
\tilde{\Gamma}_{ab} \equiv \Big( \frac{\partial h}{\partial \theta^a} \Big| \frac{\partial h}{\partial \theta^a}\Big) + \frac{1}{\sigma_{\theta^a}^2} \delta_{ab}
\end{equation}
where $h$ is the waveform template, $\sigma_{\theta^a}$ is the parameters' prior root-mean-square estimate, and the inner product is defined by:
\begin{equation}
(a|b) \equiv 2 \int^{\infty}_0\frac{\tilde{a}^*\tilde{b}+\tilde{b}^*\tilde{a}}{S_n(f)}df.
\end{equation}

In this analysis, we consider the ``PhenomD" (IMRD) waveform template~\cite{PhenomDI,PhenomDII} modified by two different tidal corrections: the 6PN tidal correction shown in Ref.~\cite{Wade:tidalCorrections} (IMRD + 6PN), and the NRTidal correction shown in Ref.~\cite{Samajdar:NRTidal} (IMRD + NRTidal).
Comparing the two tidal corrections to the IMRD waveform returns an estimate on the magnitude of waveform mismodeling systematics.
Lastly, we consider the spectral noise density $S_n^A(f)$ for 6 different interferometer designs: $A \equiv ($O2~\cite{aLIGO}, aLIGO~\cite{aLIGO}, A\texttt{+}~\cite{Ap_Voyager_CE}, Voyager~\cite{Ap_Voyager_CE}, CE~\cite{ET}, ET-D~\cite{Ap_Voyager_CE}$)$ in order to compare the statistical errors accrued on parameter extraction using future upgraded detectors.

We begin by authenticating this approach by applying a Fisher analysis (IMRD + 6PN waveform injection) to GW170817, as observed by LIGO with O2 detector sensitivity~\cite{aLIGO}.
Because only 1 event was detected, we utilize $\tilde\Lambda$ and $\delta\tilde\Lambda$ as the tidal parameters for comparison purposes.
Further, we scale the luminosity distance such that the signal-to-noise-ratio ($SNR \equiv \rho$) is fixed to be $\rho=32.4$, as was found in GW170817.
We also assume low spin priors $|\chi| \leq 0.05$, as well as $\tilde{\Lambda} \leq 3000$ and $|\delta \tilde{\Lambda}| \leq 500$~\cite{Wade:LambdaPriors}.
The resulting posterior distribution on $\tilde{\Lambda}$ has a range of $\pm 276.99$ encompassing the $90\%$ credible levels.
We compare these results with the Bayesian analysis performed in Ref.~\cite{TheLIGOScientific:2017qsa,Abbott2018}, finding close agreement with the resulting $90\%$ credible region of $70 \leq \tilde{\Lambda} \leq 720$.
This confirms the approximate validity of this method, allowing a continuation of this analysis for future detectors.
\begin{figure}
\begin{center} 
\includegraphics[width=\columnwidth]{sensitivities.eps}
\end{center}
\caption{
Square root of of the spectral noise densities $\sqrt{S_n^A(f)}$ plotted for detectors ($A$): LIGO O2 (orange), aLIGO (green), A\texttt{+} (purple), Voyager (magenta), CE (red), and ET-D (blue) as interpolated from publicly available data.
Spectral noise densities are plotted from $f_{\text{min}}=(23,10,10,7,1,1) \text{ Hz}$, respectively, to $f_{\text{max}}=1649 \text{ Hz}$.
Also shown is $2 \sqrt{f}$ multiplied by the amplitude of the PhenomD~\cite{PhenomDI,PhenomDII} waveform template.
}
\label{fig:sensitivities}
\end{figure}

Next, we consider events identical to GW170817 detected on future design sensitivity upgrades and detectors $A \equiv ($O2, aLIGO, A\texttt{+}, Voyager, CE, ET-D$)$, shown in Fig.~\ref{fig:sensitivities}.
Further, we consider the combined statistical uncertainties of $N_A$ events detected over 1 year on detector $A$, determined by integrating the local binary NS (BNS) merger rate over redshifts up to the horizon redshift of each detector.
Here we assume low spin priors $|\chi| \leq 0.05$, as well as $0 \leq \lambda_0 \leq 3207$ and $-4490 \leq \lambda_1 \leq 0$~\cite{delPozzo:TaylorTidal}\footnote{These are converted from the dimensional forms found in Ref.~\cite{delPozzo:TaylorTidal} into their corresponding dimensionless forms.}.
The process we utilize to compute single and combined statistical uncertainties for each detector sensitivity $S_n^A(f)$ is detailed in App.~\ref{app:stackingProcedure}.
From this, we determine if and when the statistical errors associated with the parameter extraction of $\lambda_0$ drop below the systematic EoS variation errors from using binary Love relations.

The results for this process upon the injection of the IMRD + 6PN waveform are compiled in Table~\ref{tab:variances}, and shown graphically in Fig.~\ref{fig:stackedFisher} where $\sigma^A_{\text{GW170817}}$ and $\sigma^A_N$ are plotted as a function of $\rho^A_{\text{GW170817}}$ (the SNR as if GW170817 were detected on future detector $A$) for 5 future detector sensitivities.
For reference, Fig.~\ref{fig:singleFisherLt} also displays the estimated $\tilde\Lambda$ extraction efficiency for \emph{one} event on each detector; as these events can not be combined like was done in the case of $\lambda_0$.
Observe how this figure shows similar results on $\tilde\Lambda$ to that found on $\lambda_0$ for the case of one event - that is, future detectors ET and CE both observe statistical errors on $\tilde\Lambda$ dropping below the systematic binary Love errors for a single event.
\red{We note that in Figs.~\ref{fig:stackedFisher} and~\ref{fig:singleFisherLt}, the single-event measurement uncertainty is slightly higher for CE than ET despite observing a larger SNR, due to the frequency-dependant detector sensitivities seen in Fig.~\ref{fig:sensitivities}.}

\begin{figure}
\begin{center} 
\includegraphics[width=\columnwidth]{singleFisherLt.eps}
\end{center}
\caption{
Estimated statistical uncertainties $\sigma^A_{\text{GW170817}}$ in the extraction of $\tilde\Lambda$ from GW170817 (O2) as if observed with future interferometers aLIGO, A\texttt{+}, Voyager, CE, and ET-D as a function of the signal-to-noise-ratio $\rho^A_{\text{GW170817}}$.
Also shown is the systematic error on $\tilde\Lambda$ derived using the same method used to compute the systematic error on $\lambda_0$.
This is similar to Fig.~\ref{fig:stackedFisher} where the extraction efficiency of $\lambda_0$ was plotted, however, in this case multiple events can \emph{not} be combined due to the variation in mass and therefore $\tilde\Lambda$ among events.
}
\label{fig:singleFisherLt}
\end{figure} 

\begin{table*}
\centering
\caption{
Tabulated results for the estimated statistical errors on the tidal deformability at $1.4\text{ M}_{\odot}$, $\lambda_0$, for the combined results of $N_A$ detections corresponding to the estimated upper, central, and lower limits of the binary NS merger detection rate $\mathcal{R}$.
This is repeated for 5 future detector sensitivities: aLIGO, A\texttt{+}, Voyager, CE, and ET.
Below, $\rho^A_{\text{GW170817}}$ and $\sigma^A_{\text{GW170817}}$ correspond to the approximate SNR and $\sigma_{\lambda_0}$ of GW170817 had it been observed by the future interferometer, and $\sigma^A_N$ corresponds to the overall uncertainty on $\lambda_0$ after $N_A$ detections on interferometer $A$.
It can be seen here that the statistical errors on $\lambda_0$ become comparable with the systematics (set to be $P_{90}=13.19$) from using improved binary Love relations from Voyager and on, indicating when it may be safe to utilize new EoS-insensitive relations.
Note that the approximation for $\sigma^A_N$ under the O2 detector sensitivity is left blank due to the singularity of the detected event.
These results are also shown graphically in Fig.~\ref{fig:stackedFisher}.
}\label{tab:variances}
\begin{tabular}{|c|@{\extracolsep{4pt}}c@{\extracolsep{0pt}}|c|@{\extracolsep{4pt}}C{1.7cm}@{\extracolsep{-2pt}}|C{1.7cm}|@{\extracolsep{-2pt}}C{1.7cm}@{\extracolsep{2pt}}|c|@{\extracolsep{0pt}}c@{\extracolsep{0pt}}|c|}
\cline{1-1}\cline{2-3}\cline{4-9}
    \multicolumn{1}{|c|}{\bfseries Detectors (A)} & \multicolumn{2}{|c|}{\bfseries GW170817} & \multicolumn{6}{|c|}{\bfseries Multiple events} \\
\cline{1-1}\cline{2-3}\cline{4-9}
\noalign{\smallskip}
\cline{2-3}\cline{4-6}\cline{7-9}
\multicolumn{1}{c}{} & \multicolumn{1}{|c|}{} & \multicolumn{1}{c|}{} & \multicolumn{3}{|c|}{} & \multicolumn{3}{|c|}{}
\\[-1em]
\multicolumn{1}{c}{}  &  \multicolumn{1}{|c|}{\multirow{2}{*}{$\rho^A_{\text{GW170817}}$}}  &  \multirow{ 2}{*}{$\sigma^A_{\text{GW170817}}$}  &  \multicolumn{3}{|c|}{$N_A$}  &  \multicolumn{3}{|c|}{$\sigma^A_N$}  \\
\cline{4-6}\cline{7-9}
\multicolumn{1}{c}{}  &  \multicolumn{1}{|c|}{}  &  \multicolumn{1}{c|}{}  &  \multicolumn{1}{|c|}{Low}  &  \multicolumn{1}{c|}{Central} &  \multicolumn{1}{c|}{High}  & \multicolumn{1}{|c|}{Low}  &  \multicolumn{1}{c|}{Central} &  \multicolumn{1}{c|}{High}\\
\cline{2-3}\cline{4-6}\cline{7-9}
\noalign{\smallskip}
\noalign{\smallskip}
\cline{1-1}\cline{2-3}\cline{4-6}\cline{7-9}
 O2  & \multicolumn{1}{|c|}{$32.40$}  & $170.13$ &  \multicolumn{1}{|c|}{--} & -- & -- & \multicolumn{1}{|c|}{--} & -- & --\\
\cline{1-1}\cline{2-3}\cline{4-6}\cline{7-9}
 aLIGO  & \multicolumn{1}{|c|}{$90.74$}  & $109.7$ &  \multicolumn{1}{|c|}{$20$} & $98$ & $304$ & \multicolumn{1}{|c|}{$182.8$} & $83.31$ & $46.95$\\
\cline{1-1}\cline{2-3}\cline{4-6}\cline{7-9}
 A\texttt{+}  & \multicolumn{1}{|c|}{$180.95$}  & $45.64$ &  \multicolumn{1}{|c|}{163} & 786 & 2,419 & \multicolumn{1}{|c|}{$58.94$} & $24.69$ & $13.66$\\
\cline{1-1}\cline{2-3}\cline{4-6}\cline{7-9}

\multicolumn{1}{|c|}{} & \multicolumn{1}{|c|}{} & \multicolumn{1}{c|}{} & \multicolumn{1}{|c|}{} & \multicolumn{1}{c|}{} & \multicolumn{1}{c|}{} & \multicolumn{1}{|c|}{} & \multicolumn{1}{c|}{} & \multicolumn{1}{c|}{}
\\[-1em]

 Voyager  & \multicolumn{1}{|c|}{$428.89$}  & $25.26$ &  \multicolumn{1}{|c|}{2,191} & 10,544 & 32,454 & \multicolumn{1}{|c|}{$20.97$} & $9.556$ & $5.323$\\
\cline{1-1}\cline{2-3}\cline{4-6}\cline{7-9}

\multicolumn{1}{|c|}{} & \multicolumn{1}{|c|}{} & \multicolumn{1}{c|}{} & \multicolumn{1}{|c|}{} & \multicolumn{1}{c|}{} & \multicolumn{1}{c|}{} & \multicolumn{1}{|c|}{} & \multicolumn{1}{c|}{} & \multicolumn{1}{c|}{}
\\[-1em]

 ET-D  & \multicolumn{1}{|c|}{$1398.03$}  & $6.871$ &  \multicolumn{1}{|c|}{71,536} & 344,270 & 1,059,638 & \multicolumn{1}{|c|}{$3.848$} & $1.732$ & $0.9621$\\
\cline{1-1}\cline{2-3}\cline{4-6}\cline{7-9}

\multicolumn{1}{|c|}{} & \multicolumn{1}{|c|}{} & \multicolumn{1}{c|}{} & \multicolumn{1}{|c|}{} & \multicolumn{1}{c|}{} & \multicolumn{1}{c|}{} & \multicolumn{1}{|c|}{} & \multicolumn{1}{c|}{} & \multicolumn{1}{c|}{}
\\[-1em]

 CE  & \multicolumn{1}{|c|}{$2807.45$}  & $7.664$ &  \multicolumn{1}{|c|}{295,200} & 1,420,650 & 4,372,653 & \multicolumn{1}{|c|}{$3.720$} & $1.668$ & $0.8979$\\
\cline{1-1}\cline{2-3}\cline{4-6}\cline{7-9}
\end{tabular}
\end{table*}

Next, we repeat the process upon the injection of the IMRD + NRTidal waveform in order to demonstrate the effect of waveform mismodeling.
The results are shown in Fig.~\ref{fig:stackedFisher} as blue outlined circles (single detection uncertainties) and the outlined dashed maroon region (combined uncertainties from multiple detections).
Observe the noticeable difference between the extraction efficiencies for the two different tidal corrections - highlighting the need for further waveform modeling for future interferometers.

Concluding, we find that the Voyager, ET, and CE detectors all exhibit enough uncertainty reduction that the systematics on $\lambda_0$ introduced from using improved EoS-insensitive relations (found to be $13.19$) no longer become negligible in the error budget.
This was also shown for the case of $\tilde\Lambda$, only taking into account single observed events, where ET and CE detectors both observe comparable statistical and systematic errors on $\tilde\Lambda$.
This implies that the use and further improvement of EoS-insensitive relations is not only justified, but necessary, for future binary NS merger detections.
Additionally, we find that the effect of waveform mismodeling is indeed important for future observations as the statistical errors continue to decrease.
Further, we investigate the dependence of the measurement accuracy on $\tilde\Lambda$ on the binary NS mass ratio $q$ for fixed chirp mass $\mathcal{M}=1.88\text{ M}_{\odot}$ in Appendix~\ref{app:measurement}.

%%%%%%%%%%%%%%%%%%%%%%%%% Begin Discussion %%%%%%%%%%%%%%%%%%%%%%%%%%%%%%%%%%%%%%%%%%%%%%%%%%%%%%%%%%%%%%%%%%%%%%%%%%%%%%%%%%%%%%%%%%%%%%%%%%%%%%%%%%%%%%%%%%%%%%%%%%%%%%%%%%%%%%%%%%%%%%%%%%%%%
\section{Conclusion and Discussion}\label{sec:conclusion}
The recent GW observation of binary NS merger GW170817 placed constraints on the supranuclear matter EoS for NSs.
We take advantage of this by generating a restricted set of spectral EoSs which agree with this observation in order to reduce the uncertainties upon the extraction of tidal parameters from future GW events.
Important previous work by Yagi and Yunes~\cite{Yagi:ILQ,Yagi:binLove} found EoS-insensitive relations between symmetric and antisymmetric combinations of NS tidal deformabilities, which aid in the extraction of said tidal parameters.
We find an improvement upon these EoS-insensitive relations by a factor of $\sim 59$\% for stars with mass ratios of $0.75$ by restricting to the EoS posterior samples from GW170817s 90\% credible region on pressure as a function of density.
Similarly, we find an increase in C-Love and I-Love-Q universality by factors of $\sim 76$\% and $\sim 50$\%, respectively.
We also show the relation between the NS radius and its tidal deformability, computed directly from the C-Love relations.
We find this relation to display a maximum error of $358.9\text{ m}$ from the fit discussed in Sec.~\ref{sec:clove}.
Additionally, we find that hybrid quark-hadron star EoSs \emph{do} obey the I-Love-Q and C-Love universality, albeit with slightly higher maximum EoS variabilities as displayed in Tabs.~\ref{tab:hybridCompare} and ~\ref{tab:hybridCompareClove}.
In addition, we find that binaries consisting of one massive hybrid quark-hadron star, and one small-mass hadron star do not obey the binary Love relations -- explained by the large difference in $\tilde{\Lambda}$ displayed between the constituent stars.
To improve on this, we construct new ``hybrid matter binary Love relations", applied to only the hybrid branches of the transitional EoSs.
This results in a maximal EoS variability similar to that found for the nuclear matter fits.
Further, we analyze the impact of this improvement for future binary NS merger detections, as the current detectors are not yet sensitive enough to accurately constrain $\tilde{\Lambda}$ for the improved EoS-insensitive relations to make a difference.
We find that future interferometers Voyager, CE, and ET-D all experience statistical uncertainties upon the extraction of $\tilde{\Lambda}$ small enough to become comparable to the systematic errors injected through the use of improved EoS-insensitive relations, as depicted in Fig.~\ref{fig:stackedFisher} and Table~\ref{tab:variances}.
Additionally, we considered the effect of waveform mismodeling by comparing Fisher analyses using two different tidal corrections to the PhenomD waveform: 6PN and NRTidal.
Doing so revealed a noticeable difference between the extraction uncertainties derived under injection of each waveform - indicating the need for further waveform modeling.
Finally, in Appendix~\ref{app:measurement} we investigate the effect of NS binary mass ratio q on the measurement accuracy of $\tilde\Lambda$.
We find that the second generation interferometers lose accuracy as one increases the mass ratio, while the third generation detectors observe a large increase in accuracy due to the correlations between $\tilde\Lambda$ and $\delta\tilde\Lambda$.
This indicates that the use and further improvement of EoS-insensitive relations is justified for future GW observations of binary NS merger events.

Future work on this subject entails an investigation into the improvement of alternative EoS insensitive relations, such as the multipole Love relations between various $l$-th order electric, magnetic, and shape tidal deformabilities as discussed in Ref.~\cite{Yagi:Multipole}.
Additionally, Lackey \emph{et al}. presented surrogate models of non-spinning effective-one-body waveform models with the the use of universal relations in Refs.~\cite{Lackey:Surrogate, Lackey:EOB}.
By reducing the number of required waveform model parameters, surrogate models aid in the extraction of NS observables from GW detections.
The improvement in the multipole love relations can then be used to increase the accuracy of surrogate models.
Additionally, we plan on a more comprehensive analysis into the intricacies of new hybrid star binary Love relations, as discussed in Sec.~\ref{sec:binary}.

%%%%%%%%%%%%%%%%%%%%%%%%% Begin Acknowledgements %%%%%%%%%%%%%%%%%%%%%%%%%%%%%%%%%%%%%%%%%%%%%%%%%%%%%%%%%%%%%%%%%%%%%%%%%%%%%%%%%%%%%%%%%%%%%%%%%%%%%%%%%%%%%%%%%%%%%%%%%%%%%%%%%%%%%%%%%%%%%%%%%%%%%
\section*{Acknowledgments}\label{acknowledgments}
K.Y. would like to acknowledge networking support by the COST Action GWverse CA16104.

%%%%%%%%%%%%%%%%%%%%%%%%% Begin Appendix A: Stacked events process %%%%%%%%%%%%%%%%%%%%%%%%%%%%%%%%%%%%%%%%%%%%%%%%%%%%%%%%%%%%%%%%%%%%%%%%%%%%%%%%%%%%%%%%%%%%%%%%%%%%%%%%%%%%%%%%%%%%%%%%%%%%%%%%%%%%%%%%%%%%%%%%%%%%%
\appendix
\section{Computation of statistical uncertainties}\label{app:stackingProcedure}

In this appendix, we detail the process used to compute the statistical uncertainties on the extraction of $\lambda_0$ from the gravitational waveform.
This is accomplished with a simple Fisher analysis described in Sec.~\ref{sec:futureObservations}, where we first consider single-events identical to GW170817 as if they were detected on future detectors $A \equiv ($ O2~\cite{aLIGO}, aLIGO~\cite{aLIGO}, A\texttt{+}~\cite{Ap_Voyager_CE}, Voyager~\cite{Ap_Voyager_CE}, ET~\cite{ET}, CE~\cite{Ap_Voyager_CE}$)$.
We conclude by simulating a population of $N_A$ events for each interferometer, approximating the number of events detected on detector $A$ over an observing period of one year.
This is used to combine the statistical uncertainties, resulting in an approximation on the overall measurement accuracy of $\lambda_0$, which is compared to the systematic errors computed in Sec.~\ref{sec:marginalization}.
The process used to achieve this is outlined below:

\begin{enumerate}
\item[(i)] Perform a Fisher analysis as outlined in Sec.~\ref{sec:futureObservations} using detector sensitivity $S_n^A(f)$, while restricting the luminosity distance $D_L$ such that an SNR of $\rho^{\text{O2}}_{\text{GW170817}}=32.4$ would be achieved on O2 sensitivity $S_n^{\text{O2}}(f)$.
Here we assume low spin priors $|\chi| \leq 0.05$, as well as $0 \leq \lambda_0 \leq 3207$ and $-4490 \leq \lambda_1 \leq 0$~\cite{delPozzo:TaylorTidal} (These are converted from their corresponding dimensional forms).
This results in a SNR $\rho^A_{\text{GW170817}}$ and a single-event statistical error $\sigma_\text{GW170817}^A$ accrued in the extraction of $\lambda_0$ on detector $A$.

\item[(ii)] Generate a population of $N_A$ events corresponding to the expected binary NS merger detection rate for detector $A$, following the probability distribution~\cite{Shutz:SNR,Chen:SNR}:
\begin{equation}\label{eq:population}
f(\rho)=3 \rho_{\text{th}}^3/\rho^4
\end{equation}
with a network SNR threshold of $\rho_{\text{th}}=8$.
The number of events $N_A$ is calculated by taking into account the BNS merger rate history throughout all redshift values within detector As' horizon distance $z_h$, as shown by Eq. (10) of Ref.~\cite{Cutler:BNSmerger}:
\begin{equation}
N_A=\Delta \tau_0 \int\limits^{z_{h}}_0 4 \pi \lbrack  a_0r_1(z)\rbrack^2 \mathcal{R} r(z) \frac{d \tau}{dz} dz.
\end{equation}
Here, $a_0r_1(z)$, $\frac{d\tau}{dz}$, and $r(z)$ for our chosen cosmology are given by:
\begin{equation}
a_0r_1(z) = \frac{1}{H_0}\int\limits^z_0 \frac{dz'}{\sqrt{(1-\Omega_{\Lambda})(1+z')^3+\Omega_{\Lambda}}},
\end{equation}
\begin{equation}
\frac{d\tau}{dz} = \frac{1}{H_0} \frac{1}{1+z}\frac{1}{\sqrt{(1-\Omega_{\Lambda})(1+z')^3+\Omega_{\Lambda}}},
\end{equation}
\begin{equation}
r(z) = \left\{
\begin{array}{ll}
      1+2z & z \leq 1 \\
      \frac{3}{4}(5-z) & 1\leq z\leq 5 \\
      0 & z\geq 5\\ 
\end{array} 
\right.
\end{equation}

where $H_0 = 70 \text{km s}^{-1}\text{Mpc}^{-1}$ is the local Hubble constant, and $\Omega_{\Lambda}=0.67$ is the universe's vacuum energy density.
Here we choose an observing period $\Delta \tau_0 = 1$ year, and calculate the detection rate for the upper, central, and lower limits of the local binary NS coalescence rate density $\mathcal{R}=1540^{+3200}_{-1220} \text{ Gpc}^{-3}\text{yr}^{-1}$~\cite{Abbott2017}, giving the rates $N_A$ shown in the second column of Table~\ref{tab:variances}, as well as the turquoise shaded region in Fig.~\ref{fig:stackedFisher}.

\item[(iii)] Compute the combined population standard deviation $\sigma_{N_A}$, taking into account sources at varying redshifts as was done in Eq. (3) of Ref.~\cite{Takahiro}:
\begin{equation}
\sigma_{N_A}^{-2}=\Delta \tau \int\limits^{z_h}_04 \pi \lbrack a_0 r_1(z)\rbrack^2 \mathcal{R}r(z)\frac{d\tau}{dz}\sigma^A_i(z)^{-2}dz.
\end{equation}
Here, we compute the populations' single-event uncertainties $\sigma_i^A(z)$ via the simple SNR scaling factor: 
\begin{equation}
\frac{\rho_i^A}{\rho_{\text{GW170817}}^A} = \frac{\sigma_{\text{GW170817}}^A}{\sigma_i^A},
\end{equation}
where $\rho_i^A$ is the SNR of simulated event $i$ computed from Eq.~\ref{eq:population}, and $\rho_{\text{GW170817}}^A$ and $\sigma_{\text{GW170817}}^A$ are the known SNR and uncertainties of the GW170817 event from step (i).
\end{enumerate}

%%%%%%%%%%%%%%%%%%%%%%%%% Begin Appendix A: measurement accuracy as a functino of q %%%%%%%%%%%%%%%%%%%%%%%%%%%%%%%%%%%%%%%%%%%%%%%%%%%%%%%%%%%%%%%%%%%%%%%%%%%%%%%%%%%%%%%%%%%%%%%%%%%%%%%%%%%%%%%%%%%%%%%%%%%%%%%%%%%%%%%%%%%%%%%%%%%%%
\section{Measurement accuracy as a function of binary mass ratio}\label{app:measurement}
In this appendix, we discuss the effects of the binary NS mass ratio $q$ on the measurement accuracy of $\tilde\Lambda$ for various interferometers. 
The top panel of Fig.~\ref{fig:massVariation} displays an approximation of the $\tilde\Lambda$ measurement accuracy as a function of increasing mass ratio $q$ (for a fixed chirp mass of $\mathcal{M}=1.188 \text{ M}_{\odot}$ corresponding to GW170817), using the Fisher analysis techniques described in Sec.~\ref{sec:observations}.
We observe two points of interest about these trends: (i) the second generation detectors (O2, aLIGO, A\texttt{+}, and Voyager) typically become less accurate as the mass ratio approaches unity, and (ii) the third generation detectors (CE and ET) observe the opposite behavior-- a fast growth in accuracy as the mass ratio increases.
For the remainder of this appendix, we attempt to explain the opposing behaviors of the second and third generation interferometers on the extraction of $\tilde\Lambda$.

\begin{figure}
\begin{center} 
\includegraphics[width=.3\textwidth]{massVariation1.eps}
\includegraphics[width=.3\textwidth]{massVariation2.eps}
\includegraphics[width=.3\textwidth]{massVariation3.eps}
\end{center}
\caption{
Measurement accuracy of $\tilde\Lambda$ as a function of increasing mass ratio $q$ for interferometer sensitivities O2, aLIGO, A\texttt{+}, Voyager, CE, and ET.
Evaluated at a fixed chirp mass of $\mathcal{M}=1.188\text{ M}_{\odot}$ corresponding to GW170817, this is repeated for three cases: (i) (top) with correlations between all template waveform parameters $\theta^a$ intact; (ii) (center) with the correlations between $\tilde\Lambda$ and $\delta\tilde\Lambda$ removed; and (iii) (bottom) with correlations between all parameters removed.
The removal of parameter correlations is approximated by setting certain Fisher matrix elements to 0, as described in Appendix.~\ref{app:measurement}.
Observe how the top panel shows strong disagreement between second and third generation detectors, while the bottom panel shows identical behavior.
This indicates that parameter correlations between $\tilde\Lambda$ and higher PN order parameter $\delta\tilde\Lambda$ is the culprit in such a disagreement.
}
\label{fig:massVariation}
\end{figure} 

We begin by considering the correlations between individual template parameters and $\tilde\Lambda$, given in Eq.~\ref{eq:template}.
A simple approximation for the amplitude of these correlations can be performed by assuming that all of the other correlations are non-existing.
Individual correlations between parameters $\tilde\Lambda$ and $\theta^a$ is achieved by setting all of the off-diagonal Fisher matrix elements to $0$, except for the $(i,j)$ components such that both $i$ and $j$ correspond to the $\tilde\Lambda$ and $\theta^a$ parameters.
We follow this process for each parameter in turn, and display the results in Fig.~\ref{fig:correlations} for each interferometer.
This figure, showing only the relative differences between correlations of different parameters $\theta^a$ and $\tilde\Lambda$, reveals a few things.
First, the coalescence time $t_c$ is highly correlated with $\tilde\Lambda$ equally for each interferometer, an unsurprising fact due to the high 4PN order at which the parameter enters the waveform.
Secondly, correlations with 6PN tidal parameter $\delta\tilde\Lambda$ shows high correlations with $\tilde\Lambda$ for only the third generation detectors.
This can be explained by the high PN order at which $\delta\tilde\Lambda$ enters the waveform, an effect that only the most sensitive of detectors can probe. 

\begin{figure}
\begin{center} 
\begin{overpic}[width=\columnwidth]{correlations.eps}
\put(121,13){\fontsize{7pt}{7pt}\selectfont $\mathcal{M}$}
\end{overpic}
\end{center}
\caption{
Measurement accuracy of $\tilde\Lambda$ upon the consideration of correlations between \emph{only} $\tilde\Lambda$ and other individual template parameters $\theta^a$ shown in Eq.~\ref{eq:template}.
This process, repeated for 6 interferometer design sensitivities O2, aLIGO, A\texttt{+}, Voyager, CE, and ET, is approximated by setting all off-diagonal Fisher matrix elements to 0 except for the $(i,j)$ components such that both $i$ and $j$ correspond to the $\tilde\Lambda$ and $\theta^a$ parameters.
Observe how correlations with 4PN parameter $t_c$ (coalescence time) remain to be highly correlated with $\tilde\Lambda$ among all 6 interferometers.
However, correlations between 6PN tidal parameter $\delta\tilde\Lambda$ and $\tilde\Lambda$ only begin to make a strong impact for third generation detectors.
This is due to the small effect that $\delta\tilde\Lambda$ has on the waveform, only to be distinguished by the most sensitive detectors.
}
\label{fig:correlations}
\end{figure} 

Following this conclusion on the difference of correlations between tidal parameters $\tilde\Lambda$ and $\delta\tilde\Lambda$ for second and third generation detectors, we attempt to describe the opposing behaviors of measurement accuracy between the two.
Shown in the center panel of Fig.~\ref{fig:massVariation}, we approximate the measurement accuracy on $\tilde\Lambda$ with the correlations between $\delta\tilde\Lambda$ and $\tilde\Lambda$ removed, through a process similar to that described above. 
Notice how the trends between detector generations become more similar -- implying that the correlations between tidal parameters may be the cause of the opposing behaviors.
This is confirmed by the bottom panel of Fig.~\ref{fig:massVariation} where all parameter correlations have been removed from the Fisher matrix.
We see that this results in identical behaviors between all 6 representative interferometers, leading us to conclude that the discrepancy between interferometers comes from the degeneracies between different parameters in the gravitational waveform, namely the 6PN tidal parameter $\delta\tilde\Lambda$ which is only detectable by the highly sensitive third generation detectors CE and ET.

%%%%%%%%%%%%%%%%%%%%%%%%% Begin Bibliography  %%%%%%%%%%%%%%%%%%%%%%%%%%%%%%%%%%%%%%%%%%%%%%%%%%%%%%%%%%%%%%%%%%%%%%%%%%%%%%%%%%%%%%%%%%%%%%%%%%%%%%%%%%%%%%%%%%%%%%%%%%%%%%%%%%%%%%%%%%%%%%%%%%%%%
\clearpage
\bibliography{Zack}
\end{document}
