\documentclass[prd,twocolumn,nofootinbib,superscriptaddress,amsmath,amssymb]{revtex4-1}
\usepackage{mathtools}
\usepackage{graphics}
\usepackage{graphicx}
\graphicspath{{./images/}}
\usepackage{dcolumn}
\usepackage{bm}
\usepackage{dsfont} 
\usepackage{amsmath,amssymb}
\usepackage{hyperref}
\usepackage{tabularx}
\usepackage{epstopdf}
\usepackage[normalem]{ulem}
\usepackage[usenames]{color}
\usepackage{multirow}
\usepackage{makecell}
\usepackage{diagbox}
\epstopdfsetup{outdir=./images/}
\allowdisplaybreaks
\hypersetup{
    colorlinks=true,
    linkcolor=blue,
    filecolor=magenta,      
    urlcolor=blue,
    citecolor=blue
}
\urlstyle{same}

\newcommand{\red}[1]{\protect\color{red} #1 \protect\color{black}}
\newcommand{\green}[1]{\protect\color{green} #1 \protect\color{black}}
\newcommand{\blue}[1]{\protect\color{blue} #1 \protect\color{black}}
\newcommand{\black}[1]{\protect\color{black} #1 \protect\color{black}}
\newcommand{\yellow}[1]{\protect\color{yellow} #1 \protect\color{black}}
\newcommand{\bra}[1]{\protect\langle #1 |}
\newcommand{\ket}[1]{| #1 \protect\rangle}
\newcommand{\braket}[2]{\protect\langle #1 | #2 \protect\rangle}
\newcommand{\expected}[1]{\protect\langle #1 \protect\rangle}
\newcommand{\R}{{\mbox{\tiny R}}}
\newcommand{\ky}[1]{\textcolor{blue}{\it{\textbf{ky: #1}}} }
\newcommand{\zc}[1]{\textcolor{red}{\it{\textbf{zc: #1}}} }
\definecolor{red(ncs)}{rgb}{0.77, 0.01, 0.2}
\newcommand{\ny}[1]{\textcolor{blue}{NY: #1} }
\newcommand{\kc}[1]{\textcolor{green}{KC: #1} }
\newcommand{\NS}{{\mbox{\tiny NS}}}

\usepackage{array}
\newcolumntype{C}[1]{>{\centering\let\newline\\\arraybackslash\hspace{0pt}}m{#1}}


\newcolumntype{C}[1]{>{\centering\arraybackslash}m{#1}}

\def\eq#1{Eq.~(\ref{eq:#1})}
\def\Eq#1{Equation~(\ref{eq:#1})}
\def\eqs#1#2{Eqs.~(\ref{eq:#1}) \& (\ref{eq:#2})}
\def\eqlist#1#2{Eqs.~(\ref{eq:#1}-\ref{eq:#2})}
\def\Eqs#1#2{Equations~(\ref{eq:#1}) \& (\ref{eq:#2})}
\def\Eqlist#1#2{Equations~(\ref{eq:#1}-\ref{eq:#2})}
\def\fig#1{Fig.\ref{fig:#1}}
\def\figs#1#2{Figs.\ref{fig:#1} \& \ref{fig:#2}}
\def\Fig#1{Figure~\ref{fig:#1}}
\def\Figs#1#2{Figures~\ref{fig:#1} \& \ref{fig:#2}}
\def\tab#1{Table~\ref{tab:#1}}
\def\sec#1{Section~\ref{sec:#1}}



\begin{document}

\title{Universal relations after GW170817}

\author{Frodo}
\affiliation{}

\author{Sam}
\affiliation{}

\author{Pippin}
\affiliation{}

\author{Merry}
\affiliation{}


\date{\today}

%%%%%%%%%%%%%%%%%%%%%%%%%%%%%%%% Begin Abstract %%%%%%%%%%%%%%%%%%%%%%%%%%%%%%%%%%%%%%%%%%%%%%%%%%%%%%%%%%%%%%%%%%%%%%%%%%%%%%%%%%%%%%%%%%%%%%%%%%%%%%%%%%%%%%%%%%%%%%%%%%%%%%%%%%%%%%%%%%%%%%%%%%%%%%%%%%%%%
\begin{abstract}

\end{abstract}

\maketitle

%%%%%%%%%%%%%%%%%%%%%%%%%%%%%%% Begin Introduction %%%%%%%%%%%%%%%%%%%%%%%%%%%%%%%%%%%%%%%%%%%%%%%%%%%%%%%%%%%%%%%%%%%%%%%%%%%%%%%%%%%%%%%%%%%%%%%%%%%%%%%%%%%%%%%%%%%%%%%%%%%%%%%%%%%%%%%%%%%%%%%%%%%%%%%%%%%%%%

\section{Introduction}\label{sec:intro}

\subsection{Executive Summary}

%%%%%%%%%%%%%%%%%%%%%%%%%%%%%%%%% Begin theory %%%%%%%%%%%%%%%%%%%%%%%%%%%%%%%%%%%%%%%%%%%%%%%%%%%%%%%%%%%%%%%%%%%%%%%%%%%%%%%%%%%%%%%%%%%%%%%%%%%%%%%%%%%%%%%%%%%%%%%%%%%%%%%%%%%%%%%%%%%%%%%%%%%%%%%%%%%%
\section{Neutron star tidal deformability}\label{sec:tidal}
We begin by reviewing how one can extract internal structure information of NSs via GW measurement.
In the presence of a neighboring tidal field $\mathcal{E}_{ij}$, such as the binary NS system found in GW170817, NSs tidally deform away from sphericity and acquire a non-vanishing quadrupole moment $Q_{ij}$ that is characterized by the \textit{tidal deformability} $\lambda$~\cite{Flanagan2008}:
This process is characterized by the \textit{tidal deformability}, $\lambda$~\cite{hinderer-love,Yagi2013}:
\begin{equation}
Q_{ij} = - \lambda \mathcal{E}_{ij}.
\end{equation}
Such tidal deformability can be made dimensionless as
\begin{equation}
\Lambda \equiv \frac{\lambda}{M^5},
\end{equation}
and can be calculated via the following expression~\cite{hinderer-love,damour-nagar,Yagi2013}:
\begin{align}
\begin{split}
\Lambda &= \frac{16}{15} (1-2\bar{C})^2[2+2\bar{C}(y_\R-1)-y_\R]\\
& \times \{2\bar{C}[6-3y_\R+3\bar{C}(5y_\R-8)]\\ 
&+4\bar{C}^3[13-11y_\R+\bar{C}(3y_\R-2)+2\bar{C}^2(1+y_\R)]\\ 
&+3(1-2\bar{C})^2[2-y_\R+2\bar{C}(y_\R-1)]\ln{(1-2\bar{C})}\}^{-1}.
\end{split}
\end{align}
Here $\bar{C} \equiv M / R$ is the stellar compactness with $M$ and $R$ representing the NS mass and radius. $y_\R \equiv y(R)$ with $y(r) \equiv r h'(r)/h(r)$, where a prime stands for taking a derivative with respect to the radial coordinate $r$. $h$ represents the quadrupolar part of the $(t,t)$ component of the metric perturbation that satisfies the following differential equation:
\begin{align}\label{eq:deq} 
\begin{split}
& h''+ \Big\{ \frac{2}{r} + \Big\lbrack \frac{2m}{r^2}+4 \pi r (p - \epsilon ) \Big\rbrack e^{\lambda} \Big\} h'\\
&+ \Big\{4 \pi \Big\lbrack 5 \epsilon + 9p + (p+ \epsilon) \frac{d \epsilon}{dp} \Big\rbrack e^{\lambda}- \frac{6}{r^2}e^{\lambda} - \Big(\frac{d \nu}{dr} \Big)^2 \Big\}h =0
\end{split}
\end{align}
with background metric coefficients $e^{\nu} = g_{tt}$ and $e^{\lambda} = (1-2m/r)^{-1} = g_{rr}$, while $p$ and $\epsilon$ represent pressure and energy density respectively.

The above differential equation can be solved as follows.
First, one needs to prepare unperturbed background solutions by choosing a specific EoS, or $p(\epsilon)$, and solve a set of Tolman-Oppenheimer-Volkoff (TOV) equations with a chosen central density (or pressure) and appropriate boundary conditions. The stellar radius is determined from $p(R)=0$ while the mass is determined by matching the interior solution to the Schwarzschild metric.
Having such solutions at hand, one then plugs them into Eq.~\eqref{eq:deq} and solve it with the boundary condition $y(0)=2$~\cite{hinderer-love}. 

Because there are two NSs in a binary, two tidal deformabilities $\Lambda_1$ and $\Lambda_2$ associated with each star enter in the gravitational waveform.
However, extracting such parameters independently is challenging due to the strong correlation between them\footnote{One way to cure this problem is to use universal relations between them~\cite{Yagi:2015pkc,Yagi:2016qmr}.}. 
Thus, what one can measure is the dominant tidal parameter in the waveform, which corresponds to the weighted average tidal deformability given by~\cite{Flanagan2008}
\begin{equation}
\tilde{\Lambda} = \frac{16}{13} \frac{(1+12q) \Lambda_1+(12+q)q^4\Lambda_2}{(1+q)^5},
\end{equation}
where $q \equiv m_2/m_1$ is the mass ratio between two stars.

%%%%%%%%%%%%%%%%%%%%%%%%%%%%%%%%% Begin EOS %%%%%%%%%%%%%%%%%%%%%%%%%%%%%%%%%%%%%%%%%%%%%%%%%%%%%%%%%%%%%%%%%%%%%%%%%%%%%%%%%%%%%%%%%%%%%%%%%%%%%%%%%%%%%%%%%%%%%%%%%%%%%%%%%%%%%%%%%%%%%%%%%%%%%%%%%%%%
\section{Spectral representations of neutron star equations of state}\label{sec:eos}
The structure of a NS and its tidal interactions in a binary system rely heavily on the underlying EoS of nuclear matter. 


%%%%%%%%%%%%%%%%%%%%%%%%% Begin universal relations %%%%%%%%%%%%%%%%%%%%%%%%%%%%%%%%%%%%%%%%%%%%%%%%%%%%%%%%%%%%%%%%%%%%%%%%%%%%%%%%%%%%%%%%%%%%%%%%%%%%%%%%%%%%%%%%%%%%%%%%%%%%%%%%%%%%%%%%%%%%%%%%%%%%%
\section{Universal relations}\label{sec:universal}

\subsection{I-Love-Q relations}\label{sec:ilq}

\subsection{Binary love relations}\label{sec:binary}

%%%%%%%%%%%%%%%%%%%%%%%%% Begin constraints %%%%%%%%%%%%%%%%%%%%%%%%%%%%%%%%%%%%%%%%%%%%%%%%%%%%%%%%%%%%%%%%%%%%%%%%%%%%%%%%%%%%%%%%%%%%%%%%%%%%%%%%%%%%%%%%%%%%%%%%%%%%%%%%%%%%%%%%%%%%%%%%%%%%%
\section{Constraints on neutron star equation of state}\label{sec:constraints}

%%%%%%%%%%%%%%%%%%%%%%%%% Begin Discussion %%%%%%%%%%%%%%%%%%%%%%%%%%%%%%%%%%%%%%%%%%%%%%%%%%%%%%%%%%%%%%%%%%%%%%%%%%%%%%%%%%%%%%%%%%%%%%%%%%%%%%%%%%%%%%%%%%%%%%%%%%%%%%%%%%%%%%%%%%%%%%%%%%%%%
\section{Conclusion and Discussion}\label{sec:conclusion}

%%%%%%%%%%%%%%%%%%%%%%%%% Begin Acknowledgements %%%%%%%%%%%%%%%%%%%%%%%%%%%%%%%%%%%%%%%%%%%%%%%%%%%%%%%%%%%%%%%%%%%%%%%%%%%%%%%%%%%%%%%%%%%%%%%%%%%%%%%%%%%%%%%%%%%%%%%%%%%%%%%%%%%%%%%%%%%%%%%%%%%%%
\section*{Acknowledgments}\label{acknowledgments}

%\appendix


%%%%%%%%%%%%%%%%%%%%%%%%% Begin Bibliography  %%%%%%%%%%%%%%%%%%%%%%%%%%%%%%%%%%%%%%%%%%%%%%%%%%%%%%%%%%%%%%%%%%%%%%%%%%%%%%%%%%%%%%%%%%%%%%%%%%%%%%%%%%%%%%%%%%%%%%%%%%%%%%%%%%%%%%%%%%%%%%%%%%%%%
%\clearpage
\bibliography{Zack}
%\bibliographystyle{IEEEtran}
%\onecolumngrid

\end{document}

